%Este trabalho está licenciado sob a Licença Creative Commons Atribuição-CompartilhaIgual 3.0 Não Adaptada. Para ver uma cópia desta licença, visite https://creativecommons.org/licenses/by-sa/3.0/ ou envie uma carta para Creative Commons, PO Box 1866, Mountain View, CA 94042, USA.

\chapter{Introdução à programação orientada a objetos}
Neste capítulo discutiremos os objetos e as classes de objetos. Objetos são semelhantes às estruturas estudadas no capítulo \ref{cap:estruturas} e provêm muito mais possibilidades. Além de variáveis internas, os objetos possuem funções internas, chamadas de métodos.

\section{Classes e objetos}

Uma classe de objetos é estrutura formada por variáveis internas e funções. Vejamos um exemplo de uma classe contendo três variáveis internas e um método.
Veja o seguinte exemplo:
\begin{verbatim}
#include <iostream>

class minha_classe_de_objetos{
	public:
		double x,y;
		int n;

		void multiplica(void);
};


void minha_classe_de_objetos::multiplica(void){
	x=y*n;
}


int main(int argn, char** argc){
	minha_classe_de_objetos objeto; 
	
	objeto.x=10;
	objeto.y=10;
	objeto.n=3;
	
	std::cout <<"objeto.x = " << objeto.x << std::endl;
	std::cout <<"objeto.y = " << objeto.y << std::endl;
	std::cout <<"objeto.n = " << objeto.n << std::endl;

	std::cout <<"objeto.multipla()" << std::endl;

	objeto.multiplica();

	std::cout <<"objeto.x = " << objeto.x << std::endl;
        return 0;
}
\end{verbatim}

Aqui foi declarada a classe chamada de "\verb|minha_classe_de_objetos|``. Dentro desta classe, há três variáveis públicas (depois discutiremos este conceito) e um método. As variáveis públicas dentro de uma objeto são acessadas da mesma forma como em uma estrutura. Os métodos precisam ser declarados dentro da classe e depois definidos indicando o nome da classe conforme exemplo.


\section{Variáveis e métodos privados}

Em C++, é possível criar variáveis e métodos privados, que são acessíveis apenas internamente. Veja o seguinte exemplo:
\begin{verbatim}
#include <iostream>

class minha_classe_de_objetos{
	public:
		double x,y;
		
		void multiplica(void);
		void grava_n(int);
		int le_n(void);

	private:

		int n;		
};


void minha_classe_de_objetos::multiplica(void){
	x=y*n;
}


void minha_classe_de_objetos::grava_n(int k){
	n=k;
}

int minha_classe_de_objetos::le_n(void){
	return n;
}




int main(int argn, char** argc){
	minha_classe_de_objetos objeto; 
	
	objeto.x=10;
	objeto.y=10;
	objeto.grava_n(3);
	
	std::cout <<"objeto.x = " << objeto.x << std::endl;
	std::cout <<"objeto.y = " << objeto.y << std::endl;
	std::cout <<"objeto.n = " << objeto.le_n() << std::endl;

	std::cout <<"objeto.multipla()" << std::endl;

	objeto.multiplica();

	std::cout <<"objeto.x = " << objeto.x << std::endl;
	return 0;
}
\end{verbatim}
Neste exemplo, foram criados dois métodos: \verb|grava_n()| e \verb|le_n()|. Estes métodos são públicos e permitem o acesso à variável privada $n$. Um tentativa de acessar diretamente a variável, por exemplo, $objeto.n=2$ geraria um erro de compilação.


\section{Método construtor}

É possível definir dentro de uma classe, um método a ser executado automticamente cada vez um objeto daquela classe é criado. Este método é chamando de construtor e recebe o mesmo nome da classe. Veja o exemplo:
\begin{verbatim}
 #include <iostream>

class minha_classe_de_objetos{
	public:
		double x,y;
		minha_classe_de_objetos(void);

	private:
		int n;		
};


minha_classe_de_objetos::minha_classe_de_objetos(void){
	std::cout<< "Um objeto da classe minha_classe_de_objetos foi criado."<<std::endl;
}




int main(int argn, char** argc){
	minha_classe_de_objetos objeto;


	return 0;
}
\end{verbatim}


