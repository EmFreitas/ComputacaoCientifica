%Este trabalho está licenciado sob a Licença Creative Commons Atribuição-CompartilhaIgual 3.0 Não Adaptada. Para ver uma cópia desta licença, visite https://creativecommons.org/licenses/by-sa/3.0/ ou envie uma carta para Creative Commons, PO Box 1866, Mountain View, CA 94042, USA.

\chapter{Variáveis globais, divisão do código e outros aspectos}

\section{Divisão do código em vários arquivos}
É possível quebrar um código em vários arquivos. Basta compilar todos os códigos juntos.
\begin{ex}\label{ex:int_dup_2}Implemente um código resolver novamente o exemplo \ref{ex:int_dup}, que consiste em calcular a integral
$$
\int_0^1f(t) dt,
$$
onde
$$
f(t)=\int_0^t \sin(t-\tau) e^{-\tau^2}d\tau.
$$
Use um arquivo para o main e outro com as funções.
\end{ex}
O primeiro arquivo foi salvo como \verb|main.c|
\begin{verbatim}
#include <stdio.h>
#include <stdlib.h>
#include <math.h>
double Boole(double (*f)(double, double),double a,double b, int N,double *param);
double integrando(double tau, double t);
double f(double t,double tau);

void main(void)
{
  int N=5;
  double I1=0,I2=0,erro=0,a=0,b=1,tol=1e-10,param[1]={0};
  I1=Boole(f,a,b,N,param);
  int cont=0;
  do
  {
  cont++;
  N=2*N-1;
  I2=Boole(f,a,b,N,param);
  erro=fabs(I2-I1);
  I1=I2;
  }while (erro>tol);
  printf("O valor da integral é %f\n",I2);
}
\end{verbatim}
O segundo arquivo foi salvo como \verb|func.c|
\begin{verbatim}
#include <math.h>
double Boole(double (*f)(double, double),double a,double b, int N,double *param)
{
  int i;
  double h=(b-a)/(4*N),Int=0,t=param[0];
  
  Int=14./45.*(*f)(a,t);
  for(i=0;i<N;i++)
  {
    Int+=64./45.*(*f)(a+h,t);
    Int+=24./45.*(*f)(a+2*h,t);
    Int+=64./45.*(*f)(a+3*h,t);
    Int+=28./45.*(*f)(a+4*h,t);
    a+=4*h;
  }
  Int-=14./45.*(*f)(a,t);
return h*Int;
}
double integrando(double tau, double t)
{
  return sin(t-tau)*exp(-tau*tau);
}
double f(double t,double tau)
{
  int N=5;
  double I1=0,I2=0,erro=0,a=0,b=t,tol=1e-10,param[1]={t};
  I1=Boole(integrando,a,b,N,param);
  int cont=0;
  do
  {
  cont++;
  N=2*N-1;
  I2=Boole(integrando,a,b,N,param);
  erro=fabs(I2-I1);
  I1=I2;
  }while (erro>tol);
  return I2;
}
\end{verbatim}
Compilamos com a linha de comando
\begin{verbatim}
gcc main.c func.c -lm
\end{verbatim}

\section{Makefile}
Imagine que dividimos o trabalho em diversos arquivos e estamos usando várias bibliotecas externas que precisam aparecer explicitamente na linha de comando para compilação. Provavelmente, estaremos interessados em simplificar o uso da linha de comando que compila o trabalho, ou seja, construir um Makefile. Na mesma pasta onde salvamos os arquivos do exemplo \ref{ex:int_dup_2}, vamos salvar um arquivo com o nome \verb|Makefile| (sem extensão) com o seguinte conteúdo:
\begin{verbatim}
main: main.c func.c 
	gcc main.c func.c -lm -oprog
\end{verbatim}
Agora, na linha de comando, digite apenas
\begin{verbatim}
make
\end{verbatim}
Execute o programa com a linha
\begin{verbatim}
./prog
\end{verbatim}



\section{Variáveis globais}
As vezes estamos interessados em trabalhar com variáveis globais, que podem ser enxergadas por qualquer função do código sem precisar ser passada como parâmetro.
\begin{ex}\label{ex:ordena}
Implemente um código com duas funções, uma que ordena os pontos de um vetor de $n$ posições e outra que coloca o valor da soma de todos os elementos de vetor de $n$ posições na posição $n+1$. Defina um vetor como variável global.
\end{ex}
\begin{verbatim}
#include <stdio.h>
#include <stdlib.h>

double *x;
int N;

void troca(double *a,double *b)
{
  double aux=*a;
  *a=*b;
  *b=aux;
}

void ordena()
{
  int i,ordena=1;
 
  do
  { 
    for (ordena=i=0;i<N-1;i++)
      if (x[i]>x[i+1])
      {
      troca(&x[i],&x[i+1]);
      ordena=1;
      }
  }while (ordena);
}

void coloca_soma_na_ultima_posicao()
{
  double soma=0;
  int i;
  for (i=0;i<N;i++) soma+=x[i];
  N++;
  x=realloc(x,N*sizeof(double));
  x[N-1]=soma;
}

void main(int argc,char **argv)
{
  N=argc-1;
  if (N==0)
  {
  printf("Entre com alguns números na linha de comando");
  return;
  }
  int i;
  x=malloc(N*sizeof(double));
  for (i=0;i<N;i++)
  {
  x[i]=atof(argv[i+1]);
  printf("x[%d]=%f\n",i,x[i]);
  }
  ordena();
  for (i=0;i<N;i++) printf("x[%d]=%f\n",i,x[i]);
  coloca_soma_na_ultima_posicao();
  for (i=0;i<N;i++) printf("x[%d]=%f\n",i,x[i]);
  ordena();
  for (i=0;i<N;i++) printf("x[%d]=%f\n",i,x[i]);
  coloca_soma_na_ultima_posicao();
  for (i=0;i<N;i++) printf("x[%d]=%f\n",i,x[i]);
}
\end{verbatim}


\section{time}
Uma forma de calcular o tempo de execução do programa é usar \verb|time| na linha de comando:
\begin{verbatim}
time ./a.out
\end{verbatim}
No entanto, se estamos interessados em estimar o tempo de processamento de um pedaço do código, podemos usar as funções da biblioteca \verb|time.h|. A bilioteca possui os tipos específicos de variáveis para trabalhar com tempo, a saber, \verb|time_t| e \verb|clock_t|. Descrevemos duas das função aqui:
\begin{itemize}
 \item \verb|clock_t clock(void)| retorna o tempo de processamento desde o início do programa em uma unidade de processamento. Basta dividir por \verb|CLOCKS_PER_SEC| para obter o valor em segundos.
 \item \verb|time_t time(time_t *timer)| salva a hora atual no formato \verb|time_t|.
 \item \verb|double difftime(time_t time1, time_t time2)| retorna em segundos a diferença entre time1 e time2.
\end{itemize}
\begin{ex}
Implemente um código que imprime o tempo para calcular cem mil vezes logarítmo de um número.
\end{ex}
\begin{verbatim}
#include <stdio.h>
#include <math.h>
#include <time.h>

void main(void)
{
  time_t tempo;
  tempo=time(NULL);
  double x;
  int i;
  for (i=0;i<100000;i++) 
  {
    x=log(i+1);
    printf("%f\n",x);
  }
  printf("%d s\n",(int) (time(NULL)-tempo));
}  
\end{verbatim}
Alternativamente,
\begin{verbatim}
#include <stdio.h>
#include <time.h>
#include <math.h>

int main(void)
{
  int i;
  double x;
  time_t inicio_t, fim_t;

  time(&inicio_t);
  for (i=0;i<100000;i++) 
  {
    x=log(i+1);
    printf("%f\n",x);
  }
  time(&fim_t);
  printf("Passaram-se %.0lf segundos\n", (double)difftime(fim_t, inicio_t));
  return 0;
}
\end{verbatim}

\section{Constante}
Quando definimos uma constante, significa que não desejamos que a variável troque de valor ao longo da rotina. Nesse caso, usamos a palabra-chave \verb|const| antes do tipo de variável. Observe o código abaixo:
\begin{verbatim}
#include <stdio.h>

void main(void)
{
  const int x=2;
  printf("%d\n",x);
  x=3;
  printf("%d\n",x);
}
\end{verbatim}
que dá um erro de compilação
\begin{verbatim}
teste.c: In function ?main?:
teste.c:7:3: error: assignment of read-only variable ?x?
   x=3;
\end{verbatim}



\section{Exercícios}
\begin{exer}
Resolva o exemplo \ref{ex:integral} novamente, mas use uma variável global para salvar os valores de $F(x)$.
\end{exer}




