%Este trabalho está licenciado sob a Licença Creative Commons Atribuição-CompartilhaIgual 3.0 Não Adaptada. Para ver uma cópia desta licença, visite https://creativecommons.org/licenses/by-sa/3.0/ ou envie uma carta para Creative Commons, PO Box 1866, Mountain View, CA 94042, USA.

\chapter{Arquivos}
\section{Escrevendo em arquivo o conteúdo da tela}
A forma mais simples de colocar o conteúdo da tela em arquivo é usar o comando \verb|>>| na linha de comando.
\begin{ex}
Escreva um programa que escreva Avila na tela e execute salvando o conteúdo em arquivo.
\end{ex}
\begin{verbatim}
#include <stdio.h>

int main(void)
{
  char nome[6]="Avila";
  printf("%s\n",nome);
}
\end{verbatim}
Execute com a linha
\begin{verbatim}
./a.out >> arquivo.txt
\end{verbatim}
Agora, abra o arquivo \verb|arquivo.txt| e olhe o conteúdo.

\section{Abertura e Fechamento de arquivos}
Até agora, os programas terminavam de ser executados e os resultados não eram salvos definitivamente em arquivo. Os arquivos são interessantes para fornecer dados de entrada para o programa e gravar resultados parciais e finais.

As operações mais comuns em arquivos são: abertura, leitura, escrita e fechamento. Abertura consiste em associar uma variável lógica ao nome do arquivo, leitura consiste em colocar em memória o conteúdo do arquivo, escrita é o caminho inverso da leitura e fechamento é a liberação da variável que associa o arquivo.

Em \verb|C| não existe um tipo de variável nativa para arquivos. O tipo de variável usada é \verb|FILE| da biblioteca \verb|<stdio.h>|. Na prática, declaramos um ponteiro para o tipo de variável \verb|FILE| tal como fazemos com outras variáveis:
\begin{verbatim}
int x;
double *y;
char ch;
FILE *fp;
\end{verbatim}

O comando para a abertura de um arquivo é o \verb|fopen|, que entra com a string contendo o nome do arquivo, outra string contendo os modos de abertura e retorna um ponteiro para \verb|FILE|:
\begin{verbatim}
FILE *fp;

fp=fopen(string com nome, string com modo de abertura);
\end{verbatim}
Os modos de abertura são os seguintes:
\begin{itemize}
 \item "r" (read): abertura de arquivo para leitura;
 \item "w" (write): abertura de arquivo para escrita;
 \item "a" (append): abertura de arquivo para acrescentar;
\end{itemize}
Além desses três modos, podemos usar um modo combinado:
\begin{itemize}
 \item "r+" (read): abertura de arquivo para leitura e escrita. Se o arquivo não existir, ele não é criado e, se ele já existir, os dados sobrescreverão os antigos;
 \item "w+" (write): abertura de arquivo para leitura e escrita. Se o arquivo não existir, ele é criado e, se ele já existir, será apagado e recriado;
 \item "a+" (append): abertura de arquivo para leitura e escrita. Se o arquivo não existir, ele é criado e, se ele já existir, os dados serão armazenados a partir do fim do arquivo.
\end{itemize}


Além disso, os arquivos podem ser abertos como binário ou texto. Os arquivos texto são aqueles que possuem símbolos interpretados por nós. Os arquivos binários em códigos não necessariamente legíveis e estão preparados para ser interpretados por um computador. Na prática, se você está interessado em salvar dados de simulação, vai preferir arquivo binário. Usamos a letra \verb|t| para configurar o arquivo como texto e a letra \verb|b| para binário. Assim, \verb|rb| é o modo de abertura para arquivo binário que permite leitura. Analogamente, \verb|wb|, \verb|ab|, \verb|r+b|, \verb|w+b| e \verb|a+b|.

Depois que você não precisa mais do arquivo aberto, você deverá fechá-lo. Para isso, você usará a função \verb|fclose(arquivo)|, que entrará um ponteiro para o arquivo e retornará um inteiro. A função \verb|fcloseall(arquivo)| fecha todos arquivos abertos.

\begin{ex}
Implemente um código para abrir um arquivo binário no modo \verb|rb|.
\end{ex}
\begin{verbatim}
#include <stdio.h>

int main(void)
{
  char nome[9]="nome.dat";
  FILE *arq;
 
  arq=fopen(nome,"r");
  if (arq==NULL) 
    printf("Não conseguiu abrir o arquivo\n");
  else 
    {
    printf("Arquivo aberto com sucesso\n");
    fclose(arq);
    }
}
\end{verbatim}
Observe que, se você não possui um arquivo com o nome \verb|nome.dat| na mesma pasta onde está seu código em C, esse programa retorna \verb|Não conseguiu abrir o arquivo|.

\begin{ex}
Implemente um código para abrir um arquivo binário no modo \verb|w+b|. Adicione o nome do arquivo em linha de comando.
\end{ex}
\begin{verbatim}
#include <stdio.h>

int main(int argc,char **argv)
{
  char *nome;
  nome=argv[1];
  FILE *arq;
 
  arq=fopen(nome,"w+b");
  if (arq==NULL) 
    printf("Não conseguiu abrir o arquivo\n");
  else 
    {
    printf("Arquivo aberto com sucesso\n");
    fclose(arq);
    }
}
\end{verbatim}
\section{Leitura e escrita de arquivos texto}
Vamos começar com o comando \verb|fputc|, que entra um caractere e um ponteiro para arquivo e retorna um inteiro. Esse comando escreve o caractere no arquivo.
\begin{ex}
Escreva o sobrenome Avila em um arquivo de texto.
\end{ex}
\begin{verbatim}
#include <stdio.h>

int main(void)
{
  FILE *arq;
 
  arq=fopen("nome.txt","wt");
  if (arq==NULL) 
    printf("Não conseguiu abrir o arquivo\n");
  else 
    {
    printf("Arquivo aberto com sucesso\n");
    fputc('A',arq);
    fputc('v',arq);
    fputc('i',arq);
    fputc('l',arq);
    fputc('a',arq);

    fclose(arq);
    }
}
\end{verbatim}
Vá até a pasta onde está nome.txt a abre a arquivo com o gedit. Para ler um caractere de um arquivo, usamos comando \verb|fgetc| que entra um ponteiro para arquivo e retorna um inteiro relacionado ao caractere pelo tabela ASCII. O retorno EOF (End-of-File), indica que não havia caractere no arquivo para ser lido. A constante EOF vale $-1$.
\begin{ex}
Escreva um programa para ler o arquivo gerado pela rotina anterior e imprime o resultado na tela.
\end{ex}
\begin{verbatim}
#include <stdio.h>

int main(void)
{
  FILE *arq;
  int ch;
 
  arq=fopen("nome.txt","rt");
  if (arq==NULL) 
    printf("Não conseguiu abrir o arquivo\n");
  else 
    {
    printf("Arquivo aberto com sucesso\n");
    while ((ch=fgetc(arq))!=EOF)
      {
      printf("Caractere na tabela ASCII=%d\n",ch);
      putchar(ch);
      printf("\n");
      }
    fclose(arq);
    }
}
\end{verbatim}

\section{Leitura e escrita de arquivos binários}
No modo binário, não existe noção de linha no arquivo, mas os arquivos são lidos e escritos em blocos. Nesse modo é possível ler e escrever um vetor inteiro de uma só vez. Essas operações para acessar um arquivo binário são chamadas de acesso direto.

Para ler e escrever um arquivo binário, usamos as funções \verb|fread| e \verb|fwrite|, respectivamente. Eles têm a seguinte sintaxe:
\begin{verbatim}
int fwrite(const void *ptr, int size, int n, FILE *arq)
int fread(const void *ptr, int size, int n, FILE *arq)
\end{verbatim}
onde
\begin{itemize}
 \item \verb|ptr| é um ponteiro de qualquer tipo apontando para o lugar da memória onde será lido ou escrito.
 \item \verb|size| é o tamanho em bytes de cada um dos elementos que queremos ler ou escrever.
 \item \verb|n| é a quantidade de elementos.
 \item \verb|arq| é o ponteiro para o arquivo que estamos lendo ou escrevendo.
\end{itemize}
Essas duas funções retorno um inteiro contendo o número de elementos escritos/lidos com sucesso.
\begin{ex}
Implemente um código para escrever e ler o nome Artur.
\end{ex}
\begin{verbatim}
#include <stdio.h>

int main(void)
{
  FILE *arq;
  char str[6]="Artur";
  char vamos_ler_aqui[6];


//escrevendo 
  arq=fopen("nome.dat","w+b");
  if (arq==NULL) 
    printf("Não conseguiu abrir o arquivo\n");
  else 
    {
    printf("Arquivo aberto com sucesso\n");
    fwrite(str,sizeof(char),5,arq);
    fclose(arq);
    }

//lendo
  arq=fopen("nome.dat","rb");
  if (arq==NULL) 
    printf("Não conseguiu abrir o arquivo\n");
  else 
    {
    printf("Arquivo aberto com sucesso\n");
    fread(vamos_ler_aqui,sizeof(char),5,arq);
    fclose(arq);
    }
  printf("Nome=%s\n",vamos_ler_aqui);
}
\end{verbatim}


\begin{ex}
Implemente um código que escreve os dez primeiros números da sequência de Fibonacci em arquivo.
\end{ex}
\begin{verbatim}
#include <stdio.h>

int main(void)
{
  FILE *arq; 
  int v[10]={0,1};
  int i;
  for (i=0;i<8;i++) v[i+2]=v[i+1]+v[i];
  arq=fopen("saida.dat","w+b");
  if (arq==NULL) 
    printf("Não conseguiu abrir o arquivo\n");
  else 
    {
    printf("Arquivo aberto com sucesso\n");
    fwrite(v,sizeof(int),10,arq);
    printf("Gravou! Vamos fechar o arquivo\n");
    fclose(arq);
    }
}
\end{verbatim}
\begin{ex}\label{ex:arq2}
Implemente um código que lê o arquivo do exemplo anterior.
\end{ex}
\begin{verbatim}
#include <stdio.h>

int main(void)
{
  FILE *arq; 
  int v[10];
  int i;
  arq=fopen("saida.dat","rb");
  if (arq==NULL) 
    printf("Não conseguiu abrir o arquivo\n");
  else 
    {
    printf("Arquivo aberto com sucesso\n");
    fread(v,sizeof(int),10,arq);
    printf("Leu! Vamos fechar o arquivo\n");
    fclose(arq);
    }
  for (i=0;i<10;i++) printf("v[%i]=%d\n",i,v[i]);
}
\end{verbatim}
Em vez de ler/escrever bloco, você pode ler/escrever um de cada vez. Observe uma versão do código \ref{ex:arq2}.
\begin{verbatim}
#include <stdio.h>

int main(void)
{
  FILE *arq; 
  int v[10];
  int i;
  arq=fopen("saida.dat","rb");
  if (arq==NULL) 
    printf("Não conseguiu abrir o arquivo\n");
  else 
    {
    printf("Arquivo aberto com sucesso\n");
    for (i=0;i<10;i++) fread(v+i,sizeof(int),1,arq);
    printf("Leu! Vamos fechar o arquivo\n");
    fclose(arq);
    }
  for (i=0;i<10;i++) printf("v[%i]=%d\n",i,v[i]);
}
\end{verbatim}
ou ainda,
\begin{verbatim}
#include <stdio.h>

int main(void)
{
  FILE *arq; 
  int v[10];
  int i;
  arq=fopen("saida.dat","rb");
  if (arq==NULL) 
    printf("Não conseguiu abrir o arquivo\n");
  else 
    {
    printf("Arquivo aberto com sucesso\n");
    for (i=0;i<10;i++) fread(&v[i],sizeof(int),1,arq);
    printf("Leu! Vamos fechar o arquivo\n");
    fclose(arq);
    }
  for (i=0;i<10;i++) printf("v[%i]=%d\n",i,v[i]);
}
\end{verbatim}
O retorno das funções \verb|fread| e \verb|fwrite| são inteiro contendo o número de itens lidos com sucesso. Observe:
\begin{verbatim}
#include <stdio.h>

int main(void)
{
  FILE *arq; 
  int v[10];
  int i;
  arq=fopen("saida.dat","rb");
  if (arq==NULL) 
    printf("Não conseguiu abrir o arquivo\n");
  else 
    {
    printf("Arquivo aberto com sucesso\n");
    printf("Leu %zu inteiros\n",fread(v,sizeof(int),10,arq));
    fclose(arq);
    }
  for (i=0;i<10;i++) printf("v[%i]=%d\n",i,v[i]);
}
\end{verbatim}

As vezes queremos ler um arquivo e não sabemos o tamanho de antemão. Nesse caso, identificamos o fim do arquivo com a função \verb|feof|, que entra um ponteiro para arquivo e retorna um valor lógico indicando se está ou não no fim do arquivo.
\begin{ex}\label{ex:le}
Implemente um código que entre na linha de comando alguns números, escreva todos eles em arquivo binário.  Em seguida, faça um programa que lê o arquivo criado e escreva todos os números na tela.
\end{ex}
Programa que escreve em arquivo alguns números digitados na linha de comando:
\begin{verbatim}
#include <stdio.h>
#include <stdlib.h>

int main(int argc, char **argv)
{
  double v[20];
  FILE *arq; 
  int i=0;
  while (argv[i+1]!=NULL) 
    {
    v[i]=atof(argv[i+1]);
    i++;
    }
  arq=fopen("saida.dat","w+b");
  if (arq==NULL) 
    printf("Não conseguiu abrir o arquivo\n");
  else 
    {
    printf("Arquivo aberto com sucesso\n");
    printf("Escreveu %zu números\n",fwrite(v,sizeof(double),argc-1,arq));
    fclose(arq);
    printf("Arquivo fechado com sucesso\n");
    }
}
\end{verbatim}
Programa que lê o arquivo anterior e escreve na tela
\begin{verbatim}
#include <stdio.h>
#include <stdlib.h>

int main(void)
{
  double v[20],*p=v;
  FILE *arq; 
  int i,ch,total=0;
  arq=fopen("saida.dat","rb");
  if (arq==NULL) 
    printf("Não conseguiu abrir o arquivo\n");
  else 
    {
    printf("Arquivo aberto com sucesso\n");  
    while ((ch=fread(p,sizeof(double),1,arq))!=0)
    {
    total+=ch;
    p++;
    }
    fclose(arq);
    printf("Arquivo fechado com sucesso\n");
    }
printf("total=%d\n",total);
for (i=0;i<total;i++) printf("%f\n",v[i]);
}
\end{verbatim}
ou, equivalentemente,
\begin{verbatim}
#include <stdio.h>
#include <stdlib.h>

int main(void)
{
  double v[20],*p=v;
  FILE *arq; 
  int i,ch,total=0;
  arq=fopen("saida.dat","rb");
  if (arq==NULL) 
    printf("Não conseguiu abrir o arquivo\n");
  else 
    {
    printf("Arquivo aberto com sucesso\n");  
    //while ((ch=fread(p,sizeof(double),1,arq))!=0)
    do
    {
    ch=fread(p,sizeof(double),1,arq);
    total+=ch;
    p++;
    }while(feof(arq)==0);
    fclose(arq);
    printf("Arquivo fechado com sucesso\n");
    }
printf("total=%d\n",total);
for (i=0;i<total;i++) printf("%f\n",v[i]);
}
\end{verbatim}

\section{Acesso direto e acesso sequencial}
Até o momento nós abrimos os arquivos e lemos sequencialmente a partir do início, isto é, fizemos acesso sequencial. Agora, pretendemos ler uma posição do arquivo sem lê-lo desde o início, isto é, pretendemos acessar diretamente um elemento. Os comandos úteis para se posicionarmos ao longo do arquivo são: \verb|ftell|, que entra um ponteiro para arquivo e retorna a posição e \verb|rewind|, que entra um arquivo e o posiciona o início. 
\begin{ex}
Reescreva um código que lê o arquivo gerado no exemplo \ref{ex:le}, depois lê o primeiro número novamente.
\end{ex}
\begin{verbatim}
#include <stdio.h>
#include <stdlib.h>

int main(void)
{
  double v[20],*p=v,first;
  FILE *arq; 
  int i,ch;
  long int total;
  arq=fopen("saida.dat","rb");
  if (arq==NULL) 
    printf("Não conseguiu abrir o arquivo\n");
  else 
    {
    printf("Arquivo aberto com sucesso\n");  
    printf("%ld\n",total=ftell(arq));
    while ((ch=fread(p,sizeof(double),1,arq))!=0)
    {
    p++;
    }
    printf("%ld\n",total=ftell(arq));
    rewind(arq);
    printf("%ld\n",ftell(arq));
    fread(&first,sizeof(double),1,arq);
    printf("%ld\n",ftell(arq));
    fclose(arq);
    printf("Arquivo fechado com sucesso\n");
    }
  printf("%ld\n",total/=8);
  for (i=0;i<total;i++) printf("%f\n",v[i]);
  printf("first=%f\n",first);
}
\end{verbatim}

A função usada para posicionamento dentro do arquivo é a 
\begin{verbatim}
int fseek(FILE *arq,long int salto, int origem)
\end{verbatim}
onde
\begin{itemize}
 \item \verb|FILE| é um ponteiro para arquivo.
 \item \verb|salto| é o número de bytes que desejamos andar (positivo para frente e negativo para trás).  
 \item \verb|origem| é o local de onde queremos saltar, podemos assumir os seguintes valores:
 \subitem \verb|SEEK_SET| salto realizado a partir da origem do arquivo.
 \subitem \verb|SEEK_CUR| salto realizado a partir da posição atual.
 \subitem \verb|SEEK_END| salto realizado a partir da fim do arquivo.
 \end{itemize}
\begin{ex}
Reescreva um código que lê o arquivo gerado no exemplo \ref{ex:le}, depois lê somente os três últimos.
\end{ex}
\begin{verbatim}
#include <stdio.h>
#include <stdlib.h>

int main(void)
{
  double v[20],*p=v,last[3];
  FILE *arq;
  int i,ch;
  long int total;
  arq=fopen("saida.dat","rb");
  if (arq==NULL) 
    printf("Não conseguiu abrir o arquivo\n");
  else 
    {
    printf("Arquivo aberto com sucesso\n");  
    printf("%ld\n",total=ftell(arq));
    while ((ch=fread(p,sizeof(double),1,arq))!=0)
    {
    p++;
    }
    printf("%ld\n",total=ftell(arq));
    fseek(arq,-24,SEEK_END);
    printf("onde estou? %ld\n",ftell(arq));
    fread(last,sizeof(double),3,arq);
    fclose(arq);
    printf("Arquivo fechado com sucesso\n");
    }
  printf("%ld\n",total/=8);
  for (i=0;i<total;i++) printf("%f\n",v[i]);
  printf("\n");
  for (i=0;i<3;i++) printf("%f\n",last[i]);
}
\end{verbatim}
\begin{ex}
Vamos resolver o exemplo \ref{ex_Jacobi_2} novamente imprimindo a solução em arquivo. Vamos usar o mesmo código, ampliando apenas as linhas que salvam em arquivo binário.
\end{ex}
\begin{verbatim}
#include <stdio.h>
#include <math.h>

#define N 101 /* dimensão do sistema*/

double norma_2(double x[N])
{
  int i;
  double norma2=x[0]*x[0]/(2*(N-1));
  for (i=1;i<N-1;i++)
  {
    norma2+=x[i]*x[i]/(N-1);
  }
  norma2+=x[N-1]*x[N-1]/(2*(N-1));
  return sqrt(norma2);	
}
void Iteracao(double u_antigo[N],double u_atual[N],double matriz[N][N], double vetor[N])
{
  double aux;
  int i,j;
  for (i=0;i<N;i++)
  {
    aux=0;
    for (j=0;j<i;j++)
    aux+=matriz[i][j]*u_antigo[j];
    for (j=i+1;j<N;j++)
    aux+=matriz[i][j]*u_antigo[j];
    u_atual[i]=(vetor[i]-aux)/matriz[i][i];
  }
}
void Jacobi(double tolerancia,double u_atual[N],double vetor[N],double matriz[N][N],double malha[N])
{

  double u_antigo[N],dif[N];
  int i,controle=3;

  //chute inicial
  for (i=0;i<N;i++)u_antigo[i]=0;

  //iteração
  while (controle)
  {
    Iteracao(u_antigo,u_atual,matriz,vetor);
    for (i=0;i<N;i++) dif[i]=u_atual[i]-u_antigo[i];
    if (norma_2(dif)<tolerancia) controle--;
    else controle=3;
    for (i=0;i<N;i++) u_antigo[i]=u_atual[i];
  }
  return;
}

double solucao_analitica(double x)
{
return -25.*(x-1.)*(x-1.)*(x-1.)*(x-1.)/3-25.*x/3+25./3.;
}

main (void)
{
  int i,j;
  double tolerancia=1e-5,solucao[N],sol_anal[N];
  double matriz[N][N];
  double vetor[N],x[N];
  //malha
  double h=1./(N-1); 
  for (i=0;i<N;i++) 
    {
    x[i]=i*h;
    sol_anal[i]=solucao_analitica(x[i]);
    }
  //matriz e vetor
  for (i=0;i<N;i++) for (j=0;j<N;j++) matriz[i][j]=0;
  matriz[0][0]=1;
  vetor[0]=0;
  for (i=1;i<N-1;i++) 
  {
    matriz[i][i]=-2;
    matriz[i][i+1]=1;
    matriz[i][i-1]=1;
    vetor[i]=-100*h*h*(x[i]-1)*(x[i]-1);
  }
  matriz[N-1][N-1]=1;
  vetor[N-1]=0;

  Jacobi(tolerancia,solucao,vetor,matriz,x);
  //erro médio
  double erro=0;
  for (i=0;i<N;i++) erro+=fabs(solucao[i]-sol_anal[i]);
  erro/=N;
  printf("erro médio = %f\n",erro);
  for (i=0;i<N;i++)     printf("u_%d=%f, u(x[%d]) = %f\n",i,solucao[i],i, sol_anal[i]);
  FILE *arq;
  arq=fopen("saida.dat","w+b");
  if (arq==NULL) 
    printf("Erro ao abrir arquivo\n");
  else
    {
    printf("Arquivo aberto\n");
    fwrite(solucao,sizeof(double),N,arq);
    fwrite(sol_anal,sizeof(double),N,arq);
    printf("Gravou\n");
    fclose(arq);
    }
}
\end{verbatim}
Observe um código em scilab para ler e fazer o gráfico das duas soluções:
\begin{verbatim}
N=101;
x=linspace(0,1,N);
arq=mopen('/home/pasta/subpasta/subsubpasta/saida.dat','rb')
Snum=mget(N,'d',arq)
Sanal=mget(N,'d',arq)
mclose(arq)
plot(x,Snum)
plot(x,Sanal,'red')
\end{verbatim}


\section{Exercícios}

\begin{exer}
Escreva um programa que copia o conteúdo de um arquivo para outro arquivo. Faça versões que lê e escreve arquivo de texto e arquivo binário.
\end{exer}
\begin{exer}
Abra o gedit e escreva o seguinte arquivo:
\begin{verbatim}
Artur Avila
O Instituto
A PPGMAp
O DMPA
\end{verbatim}
Salve o arquivo com um nome de sua preferência. Agora escreve um programa para ler o arquivo e escrever na tela o seguinte resultado:
\begin{verbatim}
linha 1: Artur Avila
linha 2: O Instituto
linha 3: A PPGMAp
linha 4: O DMPA
\end{verbatim}
\end{exer}
\begin{exer}
Faça um programa que lê um arquivo binário e escreve o menor e o maior número na tela. 
\end{exer}
\begin{exer}
Faça um programa que entra uma ou duas strings na linha de comando, sendo uma delas sempre o nome de um arquivo binário que será lido. O código vai somar os números lidos. Se entrar duas strings, uma será o nome do arquivo e a outra terá três opções: \verb|-p| soma apenas os pares, \verb|-i| soma apenas os ímpares e \verb|-t| soma todos. Se entrar apenas uma string a resposta do programa será a mesma de \verb|-t|.
\end{exer}
 
\begin{exer}
Resolva novamente o exercício \ref{exerc_14.1} usando a seguinte estratégia:
\begin{itemize}
 \item Resolva primeiro o problema linear
 $$\left\{\begin{array}{l}-u_{xx}=100,~~ 0<x<2.\\
u(0)=0\\
u(2)=10\end{array}
\right.
$$
e salve a solução em arquivo binário.
 \item Leia a solução do item anterior e use-a como chute inicial para o problema não linear
 $$\left\{\begin{array}{l}-u_{xx}=100- \frac{u^4}{10000},~~ 0<x<2.\\
u(0)=0\\
u(2)=10\end{array}
\right.
$$
Salve a solução em arquivo e use-a para fazer gráfico no scilab.
\end{itemize}
\end{exer}

\begin{exer}
Resolva novamente os exercícios \ref{edo1} e \ref{edo2}, imprima a solução em arquivo e faça o gráfico no scilab.
\end{exer}

\begin{exer}(Ajuste linear - mínimos quadrados)

Dada $m$ funções $\{f_1(x), f_2(x), \dotsc, f_m(x)\}$ e um conjunto de $n$ pares ordenados $\{(x_1, y_1)$, $(x_2, y_2)$, $\ldots$, $(x_n, y_n)\}$, o problema de mínimos quadrados linear consiste em calcular os coeficientes $a_1$, $a_2$, $\ldots$, $a_m$ tais que a função dada por
\begin{equation}
  f(x) = \sum_{j=1}^m a_jf_j(x) = a_1f_1(x)+a_2f_2(x)+\ldots+a_mf_m(x)
\end{equation}
minimiza o resíduo
\begin{equation}
  R= \sum_{i=1}^n \left[f(x_i)-y_i\right]^2.
\end{equation}
Aqui, a minimização é feita por todas as possíveis escolhas dos coeficientes $a_1$, $a_2$, $\ldots$, $a_m$. Esse problema é equivalente a encontrar a solução o sistema $Ma=w$, onde a matriz $M$ é dada por:
\begin{eqnarray*}
M = \begin{bmatrix}
\sum\limits_{i=1}^n f_1(x_i)^2 & \sum\limits_{i=1}^n f_2(x_i) f_1(x_i) & \!\cdots\! & \sum\limits_{i=1}^n f_m(x_i) f_1(x_i)\\
\sum\limits_{i=1}^n f_1(x_i) f_2(x_i)&\sum\limits_{i=1}^n f_2(x_i)^2 & \!\cdots\! & \sum\limits_{i=1}^n f_m(x_i)  f_2(x_i)\\
\sum\limits_{i=1}^n f_1(x_i) f_3(x_i)&\sum\limits_{i=1}^n f_2(x_i)f_3(x_i) & \!\cdots\! & \sum\limits_{i=1}^n f_m(x_i)  f_3(x_i)\\
\vdots & \vdots & \ddots & \vdots\\
\sum\limits_{i=1}^n f_1(x_i) f_m(x_i)&\sum\limits_{i=1}^n f_2(x_i)f_m(x_i) & \!\cdots\! & \sum\limits_{i=1}^n f_m(x_i)^2
\end{bmatrix}.
\end{eqnarray*}
e os vetores $a$ e $w$ são dados por
\begin{eqnarray*}
a=\left[
\begin{array}{c}
a_1\\
a_2\\
\vdots\\
a_m
\end{array}
\right]\qquad\text{e}\qquad w=\left[\begin{array}{c}
\sum\limits_{i=1}^n f_1(x_i) y_i\\
\sum\limits_{i=1}^n f_2(x_i) y_i\\
\sum\limits_{i=1}^n f_3(x_i) y_i\\
\vdots\\
\sum\limits_{i=1}^n f_m(x_i) y_i
\end{array}
\right].
\end{eqnarray*}
Mais simplificadamente, observamos que $M=V^TV$ e $w=V^T y$, onde a matriz $V$ é dada por:
\begin{equation}
  V=\begin{bmatrix}
    f_1(x_1)&f_2(x_1) & \cdots & f_m(x_1)\\
    f_1(x_2)&f_2(x_2) & \cdots & f_m(x_2)\\
    f_1(x_3)&f_2(x_3) & \cdots & f_m(x_3)\\
    \vdots & \vdots & \ddots & \vdots\\
    f_1(x_n)&f_2(x_n) & \cdots & f_m(x_n)
  \end{bmatrix}
\end{equation}
e $y$ é o vetor coluna $y = (y_1, y_2, \dotsc, y_N)$. Assim, o problema de ajuste se reduz a resolver o sistema linear $Ma=w$, ou $V^TVa = V^T y$. 

Implemente um código para calcular os coeficientes de uma reta que melhor se ajusta a $N$ pontos no plano. Imprime o resultado em arquivo e faça gráfico da solução no scilab.
\end{exer}
