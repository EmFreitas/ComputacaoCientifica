%Este trabalho está licenciado sob a Licença Creative Commons Atribuição-CompartilhaIgual 3.0 Não Adaptada. Para ver uma cópia desta licença, visite https://creativecommons.org/licenses/by-sa/3.0/ ou envie uma carta para Creative Commons, PO Box 1866, Mountain View, CA 94042, USA.

\chapter{Alocação dinâmica de memória}
\section{Alocação dinâmica de memória}
O objetivo é definir um ponteiro no início do programa e alocar/liberar memória durante a execução. Até agora, nossos programas alocavam memória no momento que definíamos as variáveis. Por exemplo, \verb| double x[N];| define uma variável do tipo \verb|double|, como o nome \verb|x| e aloca $8*N$ bytes para armazená-la. Suponha que nós não sabemos de antemão quantos bytes serão necessários, pois o número de bytes será calculado na primeira parte do código? Nesse caso, usamos as funções da biblioteca \verb|<stdlib.h>| para alocar/liberar memória durante a execução do programa. As funções
\begin{verbatim}
void *malloc(size_t numero_de_bytes)
\end{verbatim}
e
\begin{verbatim}
void *calloc(size_t numero_de_elementos, size_t tamanho_de_cada_elementos)
\end{verbatim}
permitem alocar memória dinamicamente. Aqui, \verb|size_t| é o tipo \verb|unsigned int| definido dentro da biblioteca \verb|<stdlib.h>|, \verb|numero_de_bytes| é o número de bytes, \verb|numero_de_elementos| é o número de elementos e \verb|tamanho_de_cada_elementos| é o tamanho de cada elementos em bytes. Ambas funções retornam um ponteiro para a posição de memória onde está sendo alocado ou \verb|NULL| no caso de falha de alocamento. A diferença básica é que a segunda função já coloca um carga inicial nula, enquanto a primeira não atribui valor algum. A função \verb|void free(void *p)| entra o ponteiro e libera a memória alocada.
\begin{ex}
Faça um programa que armazena os $N$ primeiros números da sequência de Fibonacci. Leia $N$ com a função \verb|scanf|.
\end{ex}
\begin{verbatim}
#include <stdio.h>
#include <stdlib.h>

void main(void)
{
  int N,i;
  int *FIB;
  printf("Entre com o número de termos\n");
  scanf("%d",&N);
  FIB=malloc(N*sizeof(int));
  if (FIB==NULL) printf("Não alocou memória corretamente\n");
  else
  {
    printf("Memória alocada\n");
    FIB[0]=0;
    FIB[1]=1;
    for (i=2;i<N;i++)
    {
      FIB[i]=FIB[i-1]+FIB[i-2];
    }
  }
  for (i=0;i<N;i++) printf("%d\n",FIB[i]);
  free(FIB);
  printf("Memória liberada\n");
}
\end{verbatim}

A função
\begin{verbatim}
void *realloc(void *p, size_t novo_tamanho_em_bytes)
\end{verbatim}
permite realocar a memória que já foi alocada no ponteiro \verb|p| com um novo tamanho \verb|novo_tamanho_em_bytes|.
\begin{ex}
Faça um programa que armazena os $N$ primeiros números da sequência de Fibonacci. Depois, use o realloc para armazenar novamente os $N$ números de trás para frente.
\end{ex}
\begin{verbatim}
#include <stdio.h>
#include <stdlib.h>

void main(void)
{
  int N,i;
  int *FIB;
  printf("Entre com o número de termos\n");
  scanf("%d",&N);
  FIB=malloc(N*sizeof(int));
  if (FIB==NULL) printf("Não alocou memória corretamente\n");
  else
  {
    printf("Memória alocada\n");
    FIB[0]=0;
    FIB[1]=1;
    for (i=2;i<N;i++)
    {
      FIB[i]=FIB[i-1]+FIB[i-2];
    }
    for (i=0;i<N;i++) printf("%d\n",FIB[i]);
    FIB=realloc(FIB,2*N*sizeof(double));
    for (i=N;i<2*N;i++)
    {
      FIB[i]=FIB[2*N-i-1];
    }
    for (i=0;i<2*N;i++) printf("%d\n",FIB[i]);
  }
  free(FIB);
  printf("Memória liberada\n");
}
\end{verbatim}
\begin{ex}
Voltamos ao exemplo \ref{ex_PVC_TDMA}: resolver o PVC
$$\left\{\begin{array}{l}-u_{xx}+u_x=200e^{-(x-1)^2},~~ 0<x<1.\\
15u(0)+u'(0)=500\\
10u(1)+u'(1)=1\end{array}
\right.
$$
usando o método de Thomas.
\end{ex}
\begin{verbatim}
#include <stdio.h>
#include <math.h>
#include <stdlib.h>
#define N 11

//Método de Thomas para resolver Ax=y
//x entra o vetor y e saí a solução x (N posições, i=0,....,N-1)
//a entra a subdiagonal à esquerda de A (N-1 posições, i=1,...,N-1)
//b entra a diagonal  (N posições, i=0,....,N-1)
//c entra a subdiagonal à direita de A (N-1 posições, i=0,...,N-2)
void Thomas(double x[], const double a[], const double b[], double c[]) 
{
  int i;

  /*calculo de c[0]' e d[0]'*/
  c[0] = c[0] / b[0];
  x[0] = x[0] / b[0];


  /* laço para calcular c' e d' */
  for (i = 1; i < N; i++) 
  {
    double aux = 1.0/ (b[i] - a[i] * c[i - 1]);
    c[i] = c[i] * aux;
    x[i] = (x[i] - a[i] * x[i - 1]) * aux;
  }

  /* Calculando a solução */
  for (i = N - 1; i >= 0; i--)
    x[i] = x[i] - c[i] * x[i + 1];
}


void main (void)
{
  double *x,*a,*b,*c;
  if ((x=malloc(N*sizeof(double)))==NULL)
  {
  printf("Não alocou memória");
  return;
  }
  if ((a=malloc(N*sizeof(double)))==NULL)
  {
  printf("Não alocou memória");
  return;
  }
  if ((b=malloc(N*sizeof(double)))==NULL)
  {
  printf("Não alocou memória");
  return;
  }
  if ((c=malloc(N*sizeof(double)))==NULL)
  {
  printf("Não alocou memória");
  return;
  }
  a[0]=0;c[N-1]=0;

  //malha
  int i;
  double h=1./(N-1);
  double p[N];
  for (i=0;i<N;i++) p[i]=i*h;
  
  //sistema
  b[0]=15-1/(h);
  c[0]=1/(h);
  x[0]=500;
  for (i=1;i<N-1;i++)
  {
  b[i]=2/(h*h);
  a[i]=-1/(h*h)-1/(2*h);
  c[i]=-1/(h*h)+1/(2*h);
  x[i]=200*exp(-(p[i]-1)*(p[i]-1));
  }
  b[N-1]=10+1/h;
  a[N-1]=-1/h;
  x[N-1]=1;
  Thomas(x,a,b,c);
  free(a);
  free(b);
  free(c);
  for (i=0;i<N;i++) printf("%f\n",x[i]);
  free(x);
}
\end{verbatim}
\begin{ex}\label{ex_aloc_din}
Implemente um código que aloque dinamicamente espaço para uma matriz $N\times N$ com as entradas $A_{i,j}=i^2+j^2$.
\end{ex}
\begin{verbatim}
#include <stdio.h>
#include <math.h>
#include <stdlib.h>

void preencher(double **A,int N)
{
  int i,j;
  for (i=0;i<N;i++)
  {
    for (j=0;j<N;j++)
    {
      A[i][j]=i*i+j*j;
    }
  }
}

void main (void)
{
  int i,j,N=3;
  double **A;
  if ((A=malloc(N*sizeof(double)))==NULL)
  {
  printf("Não alocou memória");
  return;
  }
  for (i=0;i<N;i++)
  {
    if ((A[i]=malloc(N*sizeof(double)))==NULL)
    {
    printf("Não alocou memória");
    return;
    }
  }
  preencher(A,N);
  for (i=0;i<N;i++)
  {
    for (j=0;j<N;j++)
    {
      printf(" %f ",A[i][j]);
    }
  printf("\n");
  }  
}
\end{verbatim}
Naturalmente, você pode armazenar toda a informação de uma matriz em um vetor. Na prática, basta a estrutura de vetor para resolver a maioria dos problemas matriciais. Observe uma versão do código do exemplo \ref{ex_aloc_din} usando apenas vetor:
\begin{verbatim}
#include <stdio.h>
#include <math.h>
#include <stdlib.h>
#define N 3

int m(int i, int j)
{
  return i+N*j;
}

void preencher(double *A)
{
  int i,j;
  for (i=0;i<N;i++)
  {
    for (j=0;j<N;j++)
    {
      A[m(i,j)]=i*i+j*j;
    }
  }
}

void main (void)
{
  int i,j;
  double *A;
  if ((A=malloc(N*N*sizeof(double)))==NULL)
  {
  printf("Não alocou memória");
  return;
  }
  preencher(A);
  for (i=0;i<N;i++)
  {
    for (j=0;j<N;j++)
    {
      printf(" %f ",A[m(i,j)]);
    }
  printf("\n");
  }  
}
\end{verbatim}


\section{Exercícios}
\begin{exer}
Implemente um programa de com as seguintes características: 
\begin{itemize}
\item lê $N$ números inteiros na linha de comando, $a_1$, $a_2$, ... $a_N$.
\item Calcule a soma $S=a_1+a_2+\cdots+a_N$.
\item Faça o seguinte teste:
\subitem Se a divisão de $S$ por 3 tiver resto 0, então aloque dinamicamente memória para o vetor de $2N$ posições com as entradas $V_{i}=a_i$, $i=1,2,...,N$ e $V_{i}=a_{2N-i}$, $i=N,N+1,...,2N$.
\subitem Se a divisão de $S$ por 3 tiver resto 1, então aloque dinamicamente memória para o vetor de $N^2$ posições com as entradas $V_{(i+N(j-1))}=a_i+a_j$, $i=1,2,...,N$ e $j=1,2,...,N$.
\subitem Se a divisão de $S$ por 3 tiver resto 2, então aloque dinamicamente memória para o vetor de $N$ posições com as entradas $V_{i}=a_i$, $i=1,2,...,N$.
\end{itemize}
\end{exer}

\begin{exer}\label{transp_1}
 Considere o seguinte problema
\begin{eqnarray*}
-\frac{dI_1}{dy}+I_1&=&\frac{1}{8}\left(I_1+I_2\right),\qquad 0<y<1\\
\frac{dI_2}{dy}+ I_2&=&\frac{1}{8}\left(I_1+I_2\right),\qquad 0<y<1\\
I_1(1)&=&1\\
I_2(0)&=&1
\end{eqnarray*}
Observações:
\begin{itemize}
 \item Embora o problema possua solução analítica, vamos resolvê-lo numericamente.
 \item Esse problema não é um problema de valor inicial, portanto, os métodos de Runge-Kutta, Adams, BDF's não se aplicam.
 \item Discretize o domínio $y$, $y_k=k/N$, $k=0,\cdots,N$, e resolva o problema
\begin{eqnarray*}
\frac{dI_1}{dy}-I_1&=&-S^{k+1/2},\qquad y_k<y<y_{k+1}\\
\frac{dI_2}{dy}+ I_2&=&S^{k+1/2},\qquad y_k<y<y_{k+1}
\end{eqnarray*}
supondo que $S^{k+1/2}$ é um valor conhecido no intervalo $[y_k,y_{k+1}]$. Assim:
\begin{eqnarray*}
\frac{d (I_1e^{-y})}{dy}&=&-e^{-y}S^{k+1/2},\qquad y_k<y<y_{k+1}\\
\frac{d(I_2e^y)}{dy}&=&e^yS^{k+1/2},\qquad y_k<y<y_{k+1}.
\end{eqnarray*}
Agora, integramos no intervalo $[y_k,y_{k+1}]$:
\begin{eqnarray*}
 \left.I_1e^{-y}\right|_{y_{k}}^{y_{k+1}}&=&S^{k+1/2}\left.e^{-y}\right|_{y_{k}}^{y_{k+1}}\\
\left.I_2e^y\right|_{y_{k}}^{y_{k+1}}&=&\left.S^{k+1/2}e^y\right|_{y_{k}}^{y_{k+1}}.
\end{eqnarray*}
Assim, temos:
\begin{eqnarray*}
 I_1^{k+1}e^{-y_{k+1}}-I_1^{k}e^{-y_{k}}&=&S^{k+1/2}\left(e^{-y_{k+1}}-e^{-y_{k}}\right)\\
 I_2^{k+1}e^{y_{k+1}}-I_2^{k}e^{y_{k}}&=&S^{k+1/2}\left(e^{y_{k+1}}-e^{y_{k}}\right).
\end{eqnarray*}
Finalmente, temos o seguinte algoritmo:
\begin{eqnarray*}
 I_1^{k}&=&I_1^{k+1}e^{y_{k}-y_{k+1}}+S^{k+1/2}\left(1-e^{y_k-y_{k+1}}\right),\qquad k=N-1,N-2,\cdots ,1,0,\\
 I_2^{k+1}&=&I_2^{k}e^{y_{k}-y_{k+1}}+S^{k+1/2}\left(1-e^{y_{k}-y_{k+1}}\right), \qquad k=0,1,2,\cdots,N-1,\\
 I_1^N&=&1,\\
 I_2^0&=&1.
\end{eqnarray*}
onde a primeira aproximação $S^{k+1/2}$ é zero e, depois de calcular $I_1^k$ e $I_2^k$, calculamos uma nova aproximação
$$
S^{k+1/2}=\frac{1}{16}\left(I_1^k+I_2^k\right)+\frac{1}{16}\left(I_1^{k+1}+I_2^{k+1}\right).
$$
\end{itemize}
Atualizamos $S^{k+1/2}$ até o processo iterativo converjir.
\end{exer}

\begin{exer}\label{transp_2}
 Resolva o seguinte problema de transporte
\begin{eqnarray*}
-\frac{dI_1}{dy}+I_1&=&\frac{1}{24}\left(I_1+2I_2+2I_3+I_4\right),\qquad 0<y<1\\
-\frac{1}{3}\frac{dI_2}{dy}+I_2&=&\frac{1}{24}\left(I_1+2I_2+2I_3+I_4\right),\qquad 0<y<1\\
\frac{1}{3}\frac{dI_3}{dy}+ I_3&=&\frac{1}{24}\left(I_1+2I_2+2I_3+I_4\right),\qquad 0<y<1\\
\frac{dI_4}{dy}+ I_4&=&\frac{1}{24}\left(I_1+2I_2+2I_3+I_4\right),\qquad 0<y<1\\
I_1(1)&=&0.5\\
I_2(1)&=&0.5\\
I_3(0)&=&0.5\\
I_4(0)&=&0.5\\
\end{eqnarray*}
Observação: Use o algoritmo introduzida do exercício \ref{transp_1}.
\end{exer}
