%Este trabalho está licenciado sob a Licença Creative Commons Atribuição-CompartilhaIgual 3.0 Não Adaptada. Para ver uma cópia desta licença, visite https://creativecommons.org/licenses/by-sa/3.0/ ou envie uma carta para Creative Commons, PO Box 1866, Mountain View, CA 94042, USA.

\chapter{Estruturas}
\section{Estruturas}
Estrutura é uma forma de agrupar um ou mais tipos de variáveis, definindo um novo tipo. Declaramos uma estrutura com a palavra \verb|struct|. Acessamos os membros de uma estrutura usando um ponto.
\begin{ex}\label{ex_struct}
Defina um estrutura que suporta uma data no formato dia/mês/ano.
\end{ex}
\begin{verbatim}
#include <stdio.h>

struct data
{
  int dia;
  char *mes;
  int ano;
}d;

void main(void)
{
  d.dia=1;
  d.mes="janeiro";
  d.ano=2018;
  printf("A data é %d de %s de %d\n",d.dia,d.mes,d.ano);
}
\end{verbatim}
No exemplo \ref{ex_struct}, nós definimos uma estrutura para armazenar a data da forma \verb|struct data|. Três variáveis existem dentro da estrutura, sendo duas inteiras e uma ponteiro para \verb|char|. Observe que \verb|data| é o nome da estrutura e \verb|d| é uma variável do tipo \verb|data|. Assim, dentro do \verb|main|, acessamos as variáveis internas da estrutura através da variavél \verb|d| usando o ponto: \verb|d.dia|, \verb|d.mes| e \verb|d.ano|.

Um estrutura admite uma carga inicial, veja outra versão do exemplo \ref{ex_struct}:
\begin{verbatim}
#include <stdio.h>

struct data
{
  int dia;
  char *mes;
  int ano;
}d={1,"janeiro",2018};

void main(void)
{
  printf("A data é %d de %s de %d\n",d.dia,d.mes,d.ano);
  d.dia=5;
  d.mes="fevereiro";
  d.ano=2017;
  printf("A data é %d de %s de %d\n",d.dia,d.mes,d.ano);
}
\end{verbatim}
A ordem da carga inicial é a mesma ordem da definição da estrutura. Alternativamente, podemos definir a variável do tipo \verb|data| no curso do programa:
\begin{verbatim}
#include <stdio.h>

struct data
{
  int dia;
  char *mes;
  int ano;
};

void main(void)
{
  struct data d={1,"janeiro",2018};
  printf("A data é %d de %s de %d\n",d.dia,d.mes,d.ano);
  d.dia=5;
  d.mes="fevereiro";
  d.ano=2017;
  printf("A data é %d de %s de %d\n",d.dia,d.mes,d.ano);
}
\end{verbatim}
Em linguagem \verb|C| também admite-se estrutura dentro de estrutura.

\section{Passagem de estrutura para funções}
A passagem de estrutura para função se faz de forma similar a outro tipo de variável.
\begin{ex}\label{ex:data}
Implemente um programa que valide uma data e imprime o resultado.
\end{ex}
\begin{verbatim}
#include <stdio.h>

struct data
{
int dia;
int mes;
int ano;
};

int bissexto(int ano);
int Max_mes(struct data d);
int testa_data(struct data d);

int main(int argc,char **argv)
{
struct data d;
d.dia=atoi(argv[1]);
d.mes=atoi(argv[2]);
d.ano=atoi(argv[3]);
if (testa_data(d)==1) printf("%d / %d / %d é uma data válida\n",d.dia,d.mes,d.ano);;
}

int bissexto(int ano)
{
if ((ano%4)==0)
{
	if ((ano%100)!=0) return 1;
}
if ((ano%400)==0) return 1;

return 0;
}
int Max_mes(struct data d)
{
 int bi=bissexto(d.ano);
 switch (d.mes)
	{
	case 1:	return 31; break;
	case 2:	
	{	
	if (bi==0) return 28;
	else  return 29;
	break;
	}
	case 3:	return 31; break;
	case 4:	return 30; break;
	case 5:	return 31; break;
	case 6:	return 30; break;
	case 7:	return 31; break;
	case 8:	return 31; break;
	case 9:	return 30; break;
	case 10:return 31; break;
	case 11:return 30; break;
	case 12:return 31; break;
	default:return 0; break;		
}

}

int testa_data(struct data d)
{
  int max_mes=Max_mes(d);
  if (max_mes==0)
  {
    printf("Escolha um mes entre 1 e 12\n");	
    return 0;
  }
  else
  {
    if (d.dia<=max_mes)
    {
      return 1;
    }
    else
    {
      printf("data incorreta: o mes %d tem no máximo %d dias\n",d.mes,max_mes);
      return 0;
    }
  }
}
\end{verbatim}
A código acima tem três função, duas delas passam uma estrutura e retornam um inteiro. A estrutura tem três membros que são variáveis do tipo \verb|int|. A função \verb|bissexto| passa a variável \verb|ano| e retorna 1 se é bissexto e 0 caso contrário. A função \verb|Max_mes| calcula quandos dias tem um mês, levando em conta se ele é bissexto. Essa função retorna 0 se o a o número de dias do mês estiver errado ou se o número do mês não estiver entre 1 e 12. 

Também podemos passar um endereço de memória de uma estrutura para uma função. Nesse caso, para pegar o valor de uma variável dentro de uma estrutura através do ponteiro, podemos usar \verb|(*p).nome| ou \verb|p->nome|.
\begin{ex}
Vamos refazer o exemplo \ref{ex_struct} usando ponteiro para estrutura.
\end{ex}
\begin{verbatim}
#include <stdio.h>

struct data
{
  int dia;
  char *mes;
  int ano;
};

void Escreve(struct data *p)
{
  printf("dia= %d\n",p->dia);
  printf("mes= %s\n",p->mes);
  printf("ano= %d\n\n",p->ano);
}


void main(void)
{
  
  struct data d;
  d.dia=1;
  d.mes="janeiro";
  d.ano=2018;
  Escreve(&d);
}
\end{verbatim}
Também podemos fazer as seguintes operações entre estruturas:
\begin{itemize}
 \item \verb|x.m| é o valor de \verb|m| dentro da estrutura \verb|x|.
 \item \verb|&x.m| é o endereço de \verb|m| dentro da estrutura \verb|x|.
 \item \verb|&x| é o endereço da estrutura \verb|x|.
 \item \verb|(*p).m| ou \verb|p->m|  é o valor de \verb|m| dentro da estrutura \verb|x| dado que \verb|p| é um ponteiro para a estrutura \verb|x|.
 \item \verb|x=y| atribui a \verb|x| todo o conteúdo de \verb|y| dado que \verb|x| e \verb|y| têm as mesmas estruturas.
 \end{itemize}
\begin{ex}
Refaça o exemplo \ref{ex:data} passando ponteiro para estrutura para as funções.
\end{ex}
\begin{verbatim}
#include <stdio.h>

struct data
{
int dia;
int mes;
int ano;
};

int bissexto(int ano);
int Max_mes(struct data *d);
int testa_data(struct data *d);

int main(int argc,char **argv)
{
if (argc<4) 
{
  printf("Entre com uma data na linha de comando\n");
  return 0;
}
struct data d;
d.dia=atoi(argv[1]);
d.mes=atoi(argv[2]);
d.ano=atoi(argv[3]);
if (testa_data(&d)==1) printf("%d / %d / %d é uma data válida\n",d.dia,d.mes,d.ano);;
}

int bissexto(int ano)
{
if ((ano%4)==0)
{
	if ((ano%100)!=0) return 1;
}
if ((ano%400)==0) return 1;

return 0;
}
int Max_mes(struct data *d)
{
 int bi=bissexto(d->ano);
 switch (d->mes)
	{
	case 1:	return 31; break;
	case 2:	
	{	
	if (bi==0) return 28;
	else  return 29;
	break;
	}
	case 3:	return 31; break;
	case 4:	return 30; break;
	case 5:	return 31; break;
	case 6:	return 30; break;
	case 7:	return 31; break;
	case 8:	return 31; break;
	case 9:	return 30; break;
	case 10:return 31; break;
	case 11:return 30; break;
	case 12:return 31; break;
	default:return 0; break;		
}

}

int testa_data(struct data *d)
{
  int max_mes=Max_mes(d);
  if (max_mes==0)
  {
    printf("Escolha um mes entre 1 e 12\n");	
    return 0;
  }
  else
  {
    if (d->dia<=max_mes)
    {
      return 1;
    }
    else
    {
      printf("data incorreta: o mes %d tem no máximo %d dias\n",d->mes,max_mes);
      return 0;
      }
    }
}
\end{verbatim}

\section{Problemas}
\begin{ex}\label{ex:integral}
Seja 
$$
F(x)=\int_0^x t^{4}e^{-50(t-2)^2}dt.
$$
Faça um código para calcular $F(x)$, salve seus valores em arquivo e visualize a função no scilab.
\end{ex}
Vamos integrar usando a regra de Boole
 \begin{eqnarray*}
  \int_a^b f(x)\, dx &\approx & \frac{2h}{45}\left(7f(x_1) + 32f(x_2)+12f(x_3)+32f(x_4)+14f(x_5)+32f(x_6)+12f(x_7)+32f(x_8)+14f(x_9)+\cdots \right.\\
  &+&\left.14f(x_{4n-3})+32f(x_{4n-2})+12f(x_{4n-1})+32f(x_{4n})+7f(x_{4n+1})\right)
\end{eqnarray*}
onde 
\begin{eqnarray*}
h &=& \frac{b-a}{4n}\\
x_i &=& a + (i-1)h, \qquad i=1,2,\cdots,4n+1.
\end{eqnarray*}
Vamos resolver o problema por partes. Primeiro, vamos implementar uma rotina que integra uma função pelo método de Boole.
\begin{verbatim}
#include <stdio.h>
#include <math.h>

double f(double t)
{
return t*t*t*t*exp(-50*(t-2)*(t-2));
}

void main(void)
{
  int N=10,i;
  double a=0,b=4,h=(b-a)/(4*N),Int=0;

  Int=14./45.*h*f(a);
  for(i=0;i<N;i++)
  {
    Int+=64./45.*h*f(a+h);
    Int+=24./45.*h*f(a+2*h);
    Int+=64./45.*h*f(a+3*h);
    Int+=28./45.*h*f(a+4*h);
    a+=4*h;
  }
  Int-=14./45.*h*f(a);
printf("%f\n",Int);
}
\end{verbatim}
Agora, vamos usar uma estrutura com $a$,$b$, e $N$.
\begin{verbatim}
#include <stdio.h>
#include <math.h>

struct INT
{
double a;
double b;
int    N;
};

double f(double t)
{
return t*t*t*t*exp(-50*(t-2)*(t-2));
}

double Boole(struct INT *I)
{
  int i,N=I->N;
  double a=I->a,b=I->b,h=(b-a)/(4*N),Int=0;

  Int=14./45.*f(a);
  for(i=0;i<N;i++)
  {
    Int+=64./45.*f(a+h);
    Int+=24./45.*f(a+2*h);
    Int+=64./45.*f(a+3*h);
    Int+=28./45.*f(a+4*h);
    a+=4*h;
  }
  Int-=14./45.*f(a);
return h*Int;
}

void main(void)
{
  struct INT I;
  I.a=0;
  I.b=4;
  I.N=10;

printf("%f\n",Boole(&I));
}
\end{verbatim}
Agora vamos fazer a primeira versão do código que resolve o problema acima. Vamos calcular $F(x)$ para onze pontos no intervalo $[0,4]$.
\begin{verbatim}
#include <stdio.h>
#include <math.h>

struct INT
{
double a;
double b;
int    N;
};

double f(double t)
{
return t*t*t*t*exp(-50*(t-2)*(t-2));
}

double Boole(struct INT *I)
{
  int i,N=I->N;
  double a=I->a,b=I->b,h=(b-a)/(4*N),Int=0;

  Int=14./45.*f(a);
  for(i=0;i<N;i++)
  {
    Int+=64./45.*f(a+h);
    Int+=24./45.*f(a+2*h);
    Int+=64./45.*f(a+3*h);
    Int+=28./45.*f(a+4*h);
    a+=4*h;
  }
  Int-=14./45.*f(a);
return h*Int;
}

void main(void)
{
  struct INT I;
  I.a=0;
  I.N=10;
  int i,M=10;
  for(i=0;i<=M;i++)
  {
  I.b=4.*i/M;
  printf("%f\n",Boole(&I));
  }
}
\end{verbatim}
Veja, até agora calculamos as integrais sem preocupação com a convergência. Então, vamos implementar um código que calcule as integrais até que o erro fica menor que uma tolerância relativa de $10^{-8}$.
\begin{verbatim}
#include <stdio.h>
#include <math.h>

struct INT
{
double a;
double b;
int    N;
};

double f(double t)
{
return t*t*t*t*exp(-50*(t-2)*(t-2));
}

double Boole(struct INT *I)
{
  int i,N=I->N;
  double a=I->a,b=I->b,h=(b-a)/(4*N),Int=0;

  Int=14./45.*f(a);
  for(i=0;i<N;i++)
  {
    Int+=64./45.*f(a+h);
    Int+=24./45.*f(a+2*h);
    Int+=64./45.*f(a+3*h);
    Int+=28./45.*f(a+4*h);
    a+=4*h;
  }
  Int-=14./45.*f(a);
return h*Int;
}
double Integral(struct INT *I)
{
  double I1,I2,aux;
  I1=Boole(I);
  do
  {
    (I->N)*=2;
    I2=Boole(I);
    aux=fabs(I2-I1);
    I1=I2;
  }while ((aux/fabs(I2))>1e-8);
  return I2;
}

double F(double x)
{
  struct INT I;
  I.a=0;
  I.N=10;
  I.b=x;
  return Integral(&I);
}
void main(void)
{
  int i,M=1000;
  double res;
  FILE *arq;
  arq=fopen("saida.dat","w+b");
  if (arq==NULL) printf("Não foi possível abrir o arquivo");
  else
  {
    printf("arquivo aberto com sucesso");
    for(i=0;i<=M;i++)
    {
      res=F(4.*i/M);
      fwrite(&res,sizeof(double),1,arq);
      printf("%f\n",res);
    }
    fclose(arq);
  }
}
\end{verbatim}
Agora vamos imprimir o resultado em arquivo e abrir no scilab:
\begin{verbatim}
#include <stdio.h>
#include <math.h>

struct INT
{
double a;
double b;
int    N;
};

double f(double t)
{
return t*t*t*t*exp(-50*(t-2)*(t-2));
}

double Boole(struct INT *I)
{
  int i,N=I->N;
  double a=I->a,b=I->b,h=(b-a)/(4*N),Int=0;

  Int=14./45.*f(a);
  for(i=0;i<N;i++)
  {
    Int+=64./45.*f(a+h);
    Int+=24./45.*f(a+2*h);
    Int+=64./45.*f(a+3*h);
    Int+=28./45.*f(a+4*h);
    a+=4*h;
  }
  Int-=14./45.*f(a);
return h*Int;
}
double Integral(struct INT *I)
{
  double I1,I2,aux;
  I1=Boole(I);
  do
  {
    (I->N)*=2;
    I2=Boole(I);
    aux=fabs(I2-I1);
    I1=I2;
  }while ((aux/fabs(I2))>1e-8);
  return I2;
}

double F(double x)
{
  struct INT I;
  I.a=0;
  I.N=10;
  I.b=x;
  return Integral(&I);
}
void main(void)
{
  int i,M=1000;
  double res;
  FILE *arq;
  arq=fopen("saida.dat","w+b");
  if (arq==NULL) printf("Não foi possível abrir o arquivo");
  else
  {
    printf("arquivo aberto com sucesso");
    for(i=0;i<=M;i++)
    {
      res=F(4.*i/M);
      fwrite(&res,sizeof(double),1,arq);
      printf("%f\n",res);
    }
    fclose(arq);
  }
}
\end{verbatim}
Abrimos no scilab com os seguintes comandos:
\begin{verbatim}
M=1000;
x=linspace(0,4,M);
arq=mopen('home/fulano/pasta1/pasta2/saida.dat','rb')
Snum=mget(M,'d',arq)
mclose(arq)
plot(x,Snum)
\end{verbatim}
Comentários sobre os códigos:
\begin{itemize}
 \item  O ponto na linha \verb|res=F(4.*i/M);| é importante. Substituir a linha por \verb|res=F(4*i/M);| não produz o resultado esperado.
 \item As linhas \verb|return Integral(&I);| e \verb|I1=Boole(I);| passam um ponteiro para a mesma estrutura \verb|INT|, mas a primera usa \verb|&I| e a segunda \verb|I|. Procure entender essa diferença na notação.
\end{itemize}

Observe que a rotina de integração vai aumentando o número de pontos até a convergência. Uma alternativa mais eficiente é implementar um esquema auto-adaptativo com refinamento local, isto é, adicionar pontos apenas nas regiões onde a integral ainda não convergiu.
\section{Exercícios}
\begin{exer}
Faça uma nova versão do exercício \ref{edo1}.
\begin{itemize}
 \item Use o método de Adams-Bashforth terceira ordem:
  \begin{equation}
  u^{(n+3)}=u^{(n+2)}+\frac{h}{12}\left[23f\left(t^{(n+2)},u(t^{(n+2)})\right)-16f\left(t^{(n+1)},u(t^{(n+1)})\right)+5f\left(t^{(n)},u(t^{(n)})\right)\right]
 \end{equation}
 \item Use o método de Kunge-Kutta terceira ordem para calcular as três primeiras iterações:
\begin{eqnarray*}
k_1&=&hf\left(t^{(n)},u^{(n)}\right)\\
k_2&=&hf\left(t^{(n)}+h/2,u^{(n)}+k_1/2\right)\\
k_3&=&hf\left(t^{(n)}+h,u^{(n)}-k1+2k_2\right)\\
u^{(n+1)}&=&u^{(n)}+\frac{k_1+4k_2+k_3}{6}
\end{eqnarray*}
\item Faça uma rotina geral para sistemas de equações diferenciais ordinárias.
\end{itemize}
Teste usando vários PVIs:
\begin{enumerate}
\item\begin{eqnarray*}
u' &=& u+e^{-t},\\
u(0)&=&1
\end{eqnarray*}
\item\begin{eqnarray*}
u' &=& -2u+v,\\
v' &=& u-2v-e^{-t^2},\\
u(0)&=&1\\
v(0)&=&0.
\end{eqnarray*}
\item\begin{eqnarray*}
u' &=& -2u+v-2w,\\
v' &=& u-2v+w,\\
w' &=& u+v-w+e^{-t},\\
u(0)&=&1\\
v(0)&=&0\\
w(0)&=&-1.
\end{eqnarray*}
\end{enumerate}
\end{exer}

\begin{exer}\label{exerc:transp_int}
Implemente um algorítmo para calcular a seguinte integral dupla:
$$
F(x)=\int_0^1\int_0^1\frac{1}{\mu}e^{-\frac{1}{\mu} |s-x|}Q(s)d\mu ds
$$
onde $Q(s)=s(1-s)$.

Dicas:
\begin{itemize}
 \item Embora a integral está bem definida, tem singularidades quando $x=s$ e $\mu=0$. Logo, um abordagem ingênua produz resultado com bastante erro.
 \item Use o seguinte truque para remover a singularidade:
 $$
F(x)=\int_0^1\int_0^1\frac{1}{\mu}e^{-\frac{1}{\mu} |s-x|}(Q(s)-Q(x))d\mu ds+Q(x)\int_0^1\int_0^1\frac{1}{\mu}e^{-\frac{1}{\mu} |s-x|}d\mu ds.
$$
\subitem A primeira integral fica numericamente bem comportada e a segunda podemos integral analiticamente:
\begin{eqnarray*}
 \int_0^1\int_0^1\frac{1}{\mu}e^{-\frac{1}{\mu} |s-x|}d\mu ds&=&\int_0^1\int_0^1\frac{1}{\mu}e^{-\frac{1}{\mu} |s-x|}ds d\mu\\
 &=&\int_0^1\left[\int_0^x\frac{1}{\mu}e^{-\frac{1}{\mu} (x-s)}ds+\int_x^1\frac{1}{\mu}e^{-\frac{1}{\mu} (s-x)}ds \right]d\mu\\ 
&=&\int_0^1\left[\left. e^{-\frac{1}{\mu} (x-s)}\right|_0^x-\left.e^{-\frac{1}{\mu} (s-x)}\right|_x^1 \right]d\mu\\ 
&=&\int_0^1\left[ 1-e^{-\frac{x}{\mu} }-e^{-\frac{1}{\mu} (1-x)}+1 \right]d\mu\\
&=&2-\int_0^1\left[ e^{-\frac{x}{\mu} }+e^{-\frac{(1-x)}{\mu} } \right]d\mu.
\end{eqnarray*}
\end{itemize}


\end{exer}