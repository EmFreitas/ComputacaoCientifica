\chapter{Funções}
\section{Funções}
Começamos com o seguinte exemplo:
\begin{ex}\label{ex13}Implemente um programa para estudar a expressão 
$$
\left(1+\frac{1}{x}\right)^x
$$
para $x=10^{2}$, $x=10^{6}$ e $x=10^{10}$.
\end{ex}
\begin{verbatim}
#include <stdio.h>
#include <math.h>
 
main (void)
  {
  double x = 1e2,y;
  y=1+1/x;
  y=pow(y,x);
  printf("(1+1/%e)^%e=%f\n",x,x,y);

  x=1e6;
  y=1+1/x;
  y=pow(y,x);
  printf("(1+1/%e)^%e=%f\n",x,x,y);

  x=1e10;
  y=1+1/x;
  y=pow(y,x);
  printf("(1+1/%e)^%e=%f\n",x,x,y);
}
\end{verbatim}
No solução do exemplo \ref{ex13}, as linhas 7, 12 e 17 são as mesmas, assim como 8, 13 e 18 e 9, 14 e 19. Para deixar o código mais sucinto e legível, convém escrever essas linhas uma única vez e chamar sua execução cada vez que desejarmos. Observe uma versão alternativa para o código do exemplo \ref{ex13}:
\begin{verbatim}
#include <stdio.h>
#include <math.h>

Euler_seq(double x)
{
  double y;
  y=1+1/x;
  y=pow(y,x);
  printf("(1+1/%e)^%e=%f\n",x,x,y);
}

main (void)
{
  Euler_seq(1e2);
  Euler_seq(1e6);
  Euler_seq(1e10);
}
\end{verbatim}
Nessa última versão, introduzimos a função \verb|Euler_seq(double x)| antes do ambiente \verb|main| que entra um parâmetro \verb|x| do tipo \verb|double| e executa as linhas de antes estavam repetidas. Dentro do ambiente \verb|main| nós chamamos a função pelo nome e dizemos qual é o \verb|double| que vamos passar.

As funções também podem retornar um valor. Observe uma outra versão para o exemplo \ref{ex13}:
\begin{verbatim}
#include <stdio.h>
#include <math.h>

double Euler_seq(double x)
{
  double y;
  y=1+1/x;
  return y=pow(y,x);
}

main (void)
{
  double z=1e2;
  printf("(1+1/%e)^%e=%f\n",z,z,Euler_seq(z));
  z=1e6;
  printf("(1+1/%e)^%e=%f\n",z,z,Euler_seq(z));
  z=1e10;
  printf("(1+1/%e)^%e=%f\n",z,z,Euler_seq(z));
}
\end{verbatim}
Seguem algumas observações sobre funções:
\begin{itemize}
 \item O nome da variável enviada para uma função não precisa ser o mesmo nome do parâmetro de entrada no cabeçalho. Observe a última versão do código do exemplo \ref{ex13}, onde a variável enviada é \verb|z| e a variável de entrada da funçao é \verb|x|. Mas observe que elas são do mesmo tipo.
 \item O cabeçalho não pode ser seguido de ponto-e-vírgula (;).
 \item Não se pode definir função dentro de outra função. No entanto, pode-se invocar uma função dentro de outra função.
 \item Depois da instrução \verb|return| pode se colocar qualquer expressão válida em C, inclusive deixar sem nada (tipo \verb|void|), desde que fique coerente com o cabeçalho.
 \item Sempre que não fique indicado o tipo de retorno, o C assume o padrão \verb|int|.
 \item As variáveis definidas dentro das funções são chamadas de variáveis locais e não podem ser chamadas de fora. Na última versão do código do exemplo \ref{ex13}, \verb|y| é uma variável local do tipo \verb|double|.
 \end{itemize}

 \begin{ex}Escreva um programa que lê dois números inteiros positivos e imprime o menor entre eles. Use a estrutura de função para retornar o menor.  
 \end{ex}
 \begin{verbatim}
#include <stdio.h>
#include <math.h>

int menor(int a,int b)
{
  if (a<=b) return a;
  else return b;
}

main (void)
{
  int x,y;
  printf("Entre com dois números inteiros positivos\n");
  scanf(" %d  %d",&x,&y);
  printf("O menor número entre eles é o %d\n",menor(x,y));
}
\end{verbatim}



\section{Problemas}
\begin{ex}\label{ex14}
Implemente um programa que aproxime o valor da integral de $f(x)$ no intervalo $[a,b]$ usando o método de Simpson composto, dado por
\begin{eqnarray*}
  \int_{a}^b f(x)\,dx &\approx& \frac{h}{3}\left[f(x_1) + 2\sum_{i=1}^{n-1} f(x_{2i+1}) + 4\sum_{i=1}^{n} f(x_{2i}) + f(x_{2n+1})\right]\\
  &=& \frac{h}{3}\left[f(x_1) + 4f(x_2)+2f(x_3)+4f(x_4)+2f(x_5)+4f(x_6)+\cdots+2f(x_{2n-1})+4f(x_{2n})+f(x_{2n+1})\right]
\end{eqnarray*}
onde 
\begin{eqnarray*}
h &=& \frac{b-a}{2n}\\
x_i &=& a + (i-1)h, \qquad i=1,2,\cdots,2n+1.
\end{eqnarray*}
\end{ex}
\begin{verbatim}
#include <stdio.h>
#include <math.h>

double f(double x)
{
  return exp(-x);
}

double Simpson(double a,double b, int n)
{
  double x,h,soma;
  int i;
  x=a;
  h=(b-a)/(2.0*n);
  soma=f(x);
  for (i=1;i<=n;i++)
  {
    x+=h;
    soma+=4*f(x);
    x+=h;
    soma+=2*f(x);
  }
  soma-=f(x);
  soma*=h/3;
  return soma;
}

main (void)
{
  printf("Integral=%f\n",Simpson(0,1,20));
}
\end{verbatim}
Nós ainda não estudamos estruturas de vetores para armazenar a malha $x_1=a$, $x_2=a+h$, $x_3=a+2h$, $\cdots$. Por isso, no código acima, calculamos dentro do \verb|for| usando a relação de recorrência $x_i=x_{i-1}+h$. Observe que calculamos $f(x_1)=f(a)$ antes do \verb|for|. No \verb|for|, nós corremos $i=1,...,n$ e a cada iteração calculamos $4f(x_i)$ e $2f(x_i+h)$. Isso significa que, na última iteração, calculamos $4f(x_{2n})$ e $2f(x_{2n+1})=2f(b)$. Mas $f(b)$ deveria ser somado uma vez só, por isso, precisamos diminuir $f(b)$ depois de encerrado o \verb|for|.

No código da solução do exemplo \ref{ex14}, nós calculamos uma aproximação para $\int_a^bf(x)dx$, mas não se preocupamos com a precisão do cálculo. Vamos introduzir o seguinte critério de parada: se o valor absoluto da diferença entre as aproximações com $n$ e $n+1$ pontos na malha for menor que $10^{-8}$, então apresenta o resultado dado pela malha com $n+1$ pontos. Façamos isso introduzindo uma nova função, como segue abaixo:
\begin{verbatim}
#include <stdio.h>
#include <math.h>

double f(double x)
{
  return exp(-x);
}

double Simpson(double a,double b, int n)
{
  double x,h,soma;
  int i;
  x=a;
  h=(b-a)/(2.0*n);
  soma=f(x);
  for (i=1;i<=n;i++)
  {
    x+=h;
    soma+=4*f(x);
    x+=h;
    soma+=2*f(x);
  }
  soma-=f(x);
  soma*=h/3;
  return soma;
}

double Integral(double a, double b,double tolerancia)
{
  int n=1;
  double I_antigo=0,I_atual=2*tolerancia;
  while ((fabs(I_antigo-I_atual)>=tolerancia))
  {
    I_antigo=Simpson(a,b,n);
    I_atual=Simpson(a,b,n+1);
    n++;
  }
  return I_atual;
}

main (void)
{
printf("Integral=%.12f\n",Integral(0,1,1e-8));
}
\end{verbatim}
Nessa versão do código, os valores \verb|I_antigo=0| e \verb|I_atual=2*tolerancia| são para que o \verb|while| execute a primeira vez. Observe que, se \verb|I_antigo=I_atual=0|, o programa não entra no \verb|while|. Uma alternativa para implementar o critério de parada é inserir uma variável de controle que, quando a tolerância é atingida, o programa executa mais algumas iterações. Isso dá uma segurança maior para o resultado obtido, evitando que, por exemplo, \verb|I_atual| e \verb|I_antigo| fiquem p?oximos, mas longe do valor da integral. Veja a próxima versão do código:
\begin{verbatim}
#include <stdio.h>
#include <math.h>

double f(double x)
{
  return exp(-x);
}

double Simpson(double a,double b, int n)
{
  double x,h,soma;
  int i;
  x=a;
  h=(b-a)/(2.0*n);
  soma=f(x);
  for (i=1;i<=n;i++)
  {
    x+=h;
    soma+=4*f(x);
    x+=h;
    soma+=2*f(x);
   }
  soma-=f(x);
  soma*=h/3;
  return soma;
}

double Integral(double a, double b,double tolerancia)
{
  int n=1,controle=2;
  double I_antigo,I_atual;
  while (controle)
  {
    I_antigo=Simpson(a,b,n);
    I_atual=Simpson(a,b,n+1);
    if((fabs(I_antigo-I_atual)<tolerancia)) controle--;
    else controle=2;	
    n++;
  }
  return I_atual;
}

main (void)
{
  printf("Integral=%.12f\n",Integral(0,1,1e-8));
}
\end{verbatim}

É importante validar a rotina antes de considerar o problema resolvido. Para isso, aproxime a integral de várias funções que conhecemos o valor exato da integral e compare.
\section{Protótipo}
Observe que, em todos os programas que implementamos até agora, a função \verb|main| é a última. Nós podemos deixar a função \verb|main| em cima, desde que definimos protótipos das funções. Observe um código anternativo para o exemplo \ref{ex_New_pont}.
\begin{verbatim}
#include <stdio.h>
#include <math.h>

double f(double x);
double df(double x);
void Newton(double x0, double tol, double *xn, int *k);

void main(void)
{
  double x0=2,xn,tol=1e-10;
  int k;
  Newton(x0,tol,&xn,&k);
  printf("Sol=%f, iterações=%d\n",xn,k);
}
double f(double x)
{
  return cos(x)-x;
}
double df(double x)
{
  return -sin(x)-1.;
}
void Newton(double x0, double tol, double *xn, int *k)
{
  double dif;
  *k=0;
  do
  {
  (*k)++;
  *xn=x0-f(x0)/df(x0);
  dif=fabs(*xn-x0);
  x0=*xn;
  }while (dif>tol);
}
\end{verbatim}


\section{Exercícios}
\begin{exer}
Escreva um código com as seguintes funções:
\begin{itemize}
 \item Calcula o máximo entre dois inteiros
 \item Testa se dois inteiros são iguais e retorna 0 ou 1.
 \item Entra dois \verb|double| e testa se um é o quadrado do outro.
 \item Entra um \verb|double| $r$ e calcula o comprimento da circunferência de raio $r$.
\end{itemize}

\end{exer}
\begin{exer}Escreva uma versão para os códigos dos exercícios \ref{exerc3.1}, \ref{exerc3.2} e \ref{exerc3.3} usando a estrutura de função.
\end{exer}
\begin{exer}Escreva uma versão para os códigos dos exercícios \ref{exerc4.1}, \ref{exerc4.2} usando a estrutura de função.
\end{exer}
\begin{exer}\label{exerc5.1}O método de Newton para calcular as raízes de uma função suave $f(x)=0$ é dado pela sequência
$$\left\{
\begin{array}{l}\displaystyle
 x_0=a,\\\\ \displaystyle
 x_{n}=x_{n-1}-\frac{f(x_{n-1})}{f'(x_{n-1})},\qquad \ n=1,2,...
\end{array}\right.,
$$
desde que $a$ seja escolhido suficientemente próximo da raíz. Implemente uma código da seguinte forma:
\begin{itemize} 
 \item Com uma função para definir $f(x)=e^{-x^2}-x$.
 \item Com uma função que entre o chute inicial $a$ e a tolerância e retorne a raíz. Coloque uma mensagens de erro no caso de não convergência.
\end{itemize}
Teste seu programa com várias funções, tais como $f(x)=\cos(10x)-e^{-x}$, $f(x)=x^4-4x^2+4$.
\end{exer}
\begin{exer}\label{exerc5.1.1}Repita o exercício \ref{exerc5.1} usando o método das Secantes para calcular as raízes da função suave $f(x)=0$. A sequência é
$$\left\{
\begin{array}{l}\displaystyle
 x_0=a,\\\\ \displaystyle
 x_2=b,\qquad b\neq a\\\\ \displaystyle
 x_{n+1}=x_{n}-f(x_n)\frac{x_{n}-x_{n-1}}{f(x_n)-f(x_{n-1})},\qquad \ n=1,2,...
\end{array}\right..
$$
Observe que aqui você precisará de dois chutes iniciais.
\end{exer}

\begin{exer}\label{exerc5.2}
Implemente um programa que aproxime o valor da integral de $f(x)$ no intervalo $[a,b]$. Faça da seguinte forma:
\begin{itemize}
 \item implemente uma função para definir $f(x)$;
 \item implemente o método de Simpson simples dado por
 $$
 \int_a^bf(x)dx\approx \frac{b-a}{6}\left(f(a)+4f\left(\frac{a+b}{2}\right)+f(b)\right)
 $$
\end{itemize}
\end{exer}

\begin{exer}
 Implemente um código similar ao dado no exemplo \ref{ex14} para aproximar $\int_a^bf(x)dx$ usando a regra dos trapézios
 \begin{equation*}
  \int_a^b f(x)\,dx \approx \frac{h}{2}\left[f(x_1) + f(x_{n+1})\right] + h\sum_{i=2}^n f(x_i),
\end{equation*}
onde 
\begin{eqnarray*}
h &=& \frac{b-a}{n}\\
x_i &=& a + (i-1)h, \qquad i=1,2,\cdots,n+1.
\end{eqnarray*}
\end{exer}
\begin{exer}
 Implemente um código similar ao dado no exemplo \ref{ex14} para aproximar $\int_a^bf(x)dx$ usando a regra de Boole
 \begin{eqnarray*}
  \int_a^b f(x)\, dx &\approx & \frac{2h}{45}\left(7f(x_1) + 32f(x_2)+12f(x_3)+32f(x_4)+14f(x_5)+32f(x_6)+12f(x_7)+32f(x_8)+14f(x_9)+\cdots \right.\\
  &+&\left.14f(x_{4n-3})+32f(x_{4n-2})+12f(x_{4n-1})+32f(x_{4n})+7f(x_{4n+1})\right)
\end{eqnarray*}
onde 
\begin{eqnarray*}
h &=& \frac{b-a}{4n}\\
x_i &=& a + (i-1)h, \qquad i=1,2,\cdots,4n+1.
\end{eqnarray*}

\end{exer}
\begin{exer}\label{exerc6.1}
 A quadrature de Gauss-Legendre com dois pontos aproxima a integral da seguinte forma
 $$
 \int_{-1}^1f(x)\, dx=f\left(-\frac{\sqrt{3}}{3}\right)+f\left(\frac{\sqrt{3}}{3}\right).
 $$
Fazendo a mudança de variável $u=\frac{2x-a-b}{b-a}$, temos:
 \begin{eqnarray*}
\int_{a}^bf(x)\,dx&=&\frac{b-a}{2}\int_{-1}^1 f\left(\frac{b-a}{2}u+\frac{a+b}{2}\right)du\\&\approx& \frac{b-a}{2}f\left(-\frac{b-a}{2}\frac{\sqrt{3}}{3}+\frac{a+b}{2}\right)+\frac{b-a}{2}f\left(\frac{b-a}{2}\frac{\sqrt{3}}{3}+\frac{a+b}{2}\right).
\end{eqnarray*}
Implemente um código que calcule uma aproximação da integral de $f(x)$ no intervalo $[a,b]$ usando a quadratura de Gauss-Legendre com dois pontos.
\end{exer}
\begin{exer}
 Implemente a seguinte regra para aproximar a integral de $f(x)$ no intervalo $[a,b]$
 \begin{itemize}
  \item Calcule a integral Gauss-Legendre com dois pontos introduzida no exercício \ref{exerc6.1}.
  \item Divida o intervalo $[a,b]$ em duas partes e use a quadratura do item anterior em cada parte.
  \item Compare as integrais dos dois itens anteriores.
  \item Divida o intervalo $[a,b]$ em quatro partes e use a mesma quadratura em cada parte.
  \item Compare as integrais com duas e quatro divisões.
 \item Repita até a convergência com tolerância de $10^{-8}$.
  \end{itemize}

\end{exer}

\begin{exer}Implemente o método da bisseção para calcular a raíz de uma função.

O método da bisseção explora o fato de que uma função contínua $f:[a, b]\to \mathbb{R}$ com $f(a)\cdot f(b) < 0$ tem um zero no intervalo $(a, b)$ (veja o teorema de Bolzano). Assim, a ideia para aproximar o zero de uma tal função $f(x)$ é tomar, como primeira aproximação, o ponto médio do intervalo $[a, b]$, isto é:
\begin{equation*}
  x^{(0)} = \frac{(a + b)}{2}. 
\end{equation*}
Pode ocorrer de $f(x^{(0)}) = 0$ e, neste caso, o zero de $f(x)$ é $x^* = x^{(0)}$. Caso contrário, se $f(a)\cdot f(x^{(0)}) < 0$, então $x^*\in (a, x^{(0)})$. Neste caso, tomamos como segunda aproximação do zero de $f(x)$ o ponto médio do intervalo $[a, x^{(0)}]$, isto é, $x^{(1)} = (a + x^{(0)})/2$. Noutro caso, temos $f(x^{(0)})\cdot f(b) < 0$ e, então, tomamos $x^{(1)} = (x^{(0)} + b)/2$. Repetimos este procedimento até alcançar a tolerância
\begin{equation*}
  \displaystyle \frac{|b^{(n)} - a^{(n)}|}{2} < TOL.
\end{equation*}
Teste usando algumas funções, tais como $f(x)=\cos(x)-x$ no intervalo $[0,2]$ ou $f(x)=e^{-x}-x$ no intervalo $[0,1]$.
\end{exer}
