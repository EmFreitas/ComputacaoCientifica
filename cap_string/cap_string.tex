%Este trabalho está licenciado sob a Licença Creative Commons Atribuição-CompartilhaIgual 3.0 Não Adaptada. Para ver uma cópia desta licença, visite https://creativecommons.org/licenses/by-sa/3.0/ ou envie uma carta para Creative Commons, PO Box 1866, Mountain View, CA 94042, USA.

\chapter{Strings}
\section{Strings}
Strings são vetores de \verb|char|.
\begin{ex}Alguns testes com strings.
\end{ex}
\begin{verbatim}
#include <stdio.h>
#include <math.h>


main (void)
{
  char nome[7]="Artur";
  char sobrenome[10]={'A','v','i','l','a'};

  printf("O nome do matemático é %s %s\n",nome,sobrenome);
  printf("\n");
  int i;
  for(i=0;i<5;i++) printf("%c\n",nome[i]);
  printf("\n");
  for(i=0;i<5;i++) printf("%c\n",sobrenome[i]);
}
\end{verbatim}

Observe que usamos \verb|%s| para imprimir um string e \verb|%c| para imprimir um caractere.

\begin{ex}Faça um programa que lê nome e o sobrenome e imprime-os.
\end{ex}
\begin{verbatim}
#include <stdio.h>
#include <math.h>


main (void)
{
  char nome[20],sobrenome[20];

  printf("Digite seu nome\n");
  scanf("%s",nome);
  printf("Digite seu sobrenome\n");
  scanf("%s",sobrenome);
  printf("\n%s",nome);
  printf(" %s\n",sobrenome);
}
\end{verbatim}
Existem várias função para manipulação de strings, tais como
\begin{itemize}
\item \verb|strlen(string)| conta o número de caracteres e retorna um inteiro positivo.
\item \verb|strcpy(string1,string2)| copia string2 para string1.
\item \verb|strcat(string1,string2)| coloca string2 imediatamente após string1.
\item \verb|strcmp(string1,string2)| compara as strings alfabeticamente e retorna verdadeiro ou falso (1 ou 0).
\end{itemize}
Essas funções não são nativas em C e precisam da biblioteca \verb|string.h| para funcionar.
\begin{ex}Faça um programa que lê nome e o sobrenome e imprime o número de letras de cada um.
\end{ex}
\begin{verbatim}
#include <stdio.h>
#include <math.h>
#include <string.h>

main (void)
{
  char nome[20],sobrenome[20];

  printf("Digite seu nome\n");
  scanf("%s",nome);
  printf("Digite seu sobrenome\n");
  scanf("%s",sobrenome);
  printf("\n%s",nome);
  printf(" %s\n",sobrenome);
  printf("Seu nome tem %d letras\n",(int) strlen(nome));
  printf("Seu sobrenome tem %d letras\n",(int) strlen(sobrenome));
}
\end{verbatim}
\begin{ex}Escreva um código que lê uma string e inverte os caracteres.
\end{ex}
\begin{verbatim}
#include <stdio.h>
#include <string.h>

char *strinv(char *s)
{
  int i,j,k=strlen(s);
  char aux;
  for (i=0,j=k-1;i<j;i++,j--)
    {
    aux=s[i];
    s[i]=s[j];
    s[j]=aux;
    }
  return s;
}

int main(void)
{
  char nome[6]="Artur";
  strinv(nome);
  printf("O nome invertido é %s\n",nome);
}
\end{verbatim}

\begin{ex}
Escreva um programa que lê a idade de um gato e escreve-o por extenso. Suponha que o gato viva no máximo 39 anos. 
\end{ex}
\begin{verbatim}
#include <stdio.h>
#include <math.h>
#include <string.h>

int unidades(char idade[],char resposta[])
{
  if (idade[0]=='0') {strcpy(resposta,"zero"); return 1;}
  if (idade[0]=='1') {strcpy(resposta,"um"); return 1;}
  if (idade[0]=='2') {strcpy(resposta,"dois"); return 1;}
  if (idade[0]=='3') {strcpy(resposta,"três"); return 1;}
  if (idade[0]=='4') {strcpy(resposta,"quatro"); return 1;}
  if (idade[0]=='5') {strcpy(resposta,"cinco"); return 1;}
  if (idade[0]=='6') {strcpy(resposta,"seis"); return 1;}
  if (idade[0]=='7') {strcpy(resposta,"sete"); return 1;}
  if (idade[0]=='8') {strcpy(resposta,"oito"); return 1;}
  if (idade[0]=='9') {strcpy(resposta,"nove"); return 1;}
  return 0;
}
int dezenas(char idade[],char resposta[])
{
  if (strcmp(idade,"10")==0) {strcpy(resposta,"dez"); return 1;}
  if (strcmp(idade,"11")==0) {strcpy(resposta,"onze"); return 1;}
  if (strcmp(idade,"12")==0) {strcpy(resposta,"doze"); return 1;}
  if (strcmp(idade,"13")==0) {strcpy(resposta,"treze"); return 1;}
  if (strcmp(idade,"14")==0) {strcpy(resposta,"quatorze"); return 1;}
  if (strcmp(idade,"15")==0) {strcpy(resposta,"quize"); return 1;}
  if (strcmp(idade,"16")==0) {strcpy(resposta,"dezesseis"); return 1;}
  if (strcmp(idade,"17")==0) {strcpy(resposta,"dezessete"); return 1;}
  if (strcmp(idade,"18")==0) {strcpy(resposta,"dezoito"); return 1;}
  if (strcmp(idade,"19")==0) {strcpy(resposta,"dezenove"); return 1;}
  if (idade[0]=='0') return 0;
  if (idade[0]=='2') 
  {
    if (strcmp(idade,"20")==0) {strcpy(resposta,"vinte"); return 1;}
    char unidade[2]={idade[1]},resp[10];
    strcpy(resposta,"vinte e ");
    if (unidades(unidade,resp)==1)
    {
    strcat(resposta,resp);    
    return 1;
    }
  }
  if (idade[0]=='3') 
  {
    if (strcmp(idade,"30")==0) {strcpy(resposta,"trinta"); return 1;}
    char unidade[2]={idade[1]},resp[10];
    strcpy(resposta,"trinta e ");
    if (unidades(unidade,resp)==1)
    {
    strcat(resposta,resp);    
    return 1;
    }
  }
   



  return 0;
}


int main (void)
{
  char idade[4]="",resposta[30]="";

  printf("Digite a idade do gato\n");
  scanf("%s",idade);
  if (strlen(idade)==1)
  {
    if (unidades(idade,resposta)==1)
    {    
      if ((idade[0]=='1')||(idade[0]=='0')) {printf("A idade é %s ano\n",resposta); return 0;}
      else {printf("A idade é %s anos\n",resposta); return 0;}
    }
    else    
    {
    printf("Idade inválida\n"); 
    return 0;
    }
  }
  if (strlen(idade)==2)
  {
    if (dezenas(idade,resposta)==1)
    {    
    printf("A idade é %s anos\n",resposta); 
    return 0;
    }
    else    
    {
    printf("Idade inválida\n"); 
    return 0;
    }
  }
printf("Idade inválida\n"); 
}
\end{verbatim}

\section{Exercícios}
\begin{exer}Escreva um programa que lê uma string \verb|s| e conta quantas vezes aparece um caractere \verb|ch|: \verb|int strcount(char *s, char ch)|.
\end{exer}
\begin{exer}Escreva um programa que lê a idade de uma pessoa e escreve-o por extenso. Suponha que uma pessoa viva no máximo 129 anos. 
\end{exer}
\begin{exer}Escreva um programa que lê dois números correspondente ao valor de um cheque (reais e centavos) e escreve-o por extenso. Suponha que o valor máximo de um cheque seja $R\$9999,99$. 
\end{exer}